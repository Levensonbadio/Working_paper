\documentclass[12pt]{article}
\usepackage{authblk}
\usepackage[english]{babel}
\usepackage{threeparttable}
% Bibliography Style
\usepackage[authoryear]{natbib}  % Enables Chicago author-date citations
\bibliographystyle{chicago}
\newcommand{\sym}[1]{\rlap{$^{#1}$}}
\usepackage{soul}
\usepackage{amsmath}
\usepackage{graphicx}   % Required for \resizebox{}
\usepackage[colorlinks=true, allcolors=blue]{hyperref}
\usepackage{ulem} % For \uline command

 \newcommand\citeapos[1]{\citeauthor{#1}'s (\citeyear{#1})} % use apostrofe for references e.g. Lehr's (1990)
\makeatletter
 \def\@mb@citenamelist{cite,citep,citet,citealp,citealt,citeapos} % this will add the custom made command 'citeapos' to the list of newcite commands
 \makeatother
% Margins: 1-inch all around
\usepackage[margin=1in]{geometry}

% Font: Times New Roman (or closest equivalent)
% \usepackage{times}  

% Table & Formatting Packages
\usepackage{threeparttable} % For notes below tables
\usepackage{multirow}
\usepackage{makecell}
\usepackage{array}          % Improved column alignment
\usepackage{siunitx}        % Proper numeric alignment
\usepackage{booktabs}       % Professional table formatting
\usepackage{float}          % Ensures float placement control

% Caption Formatting
\usepackage{caption}   
\captionsetup[table]{
  labelsep = period, % "Table 1."
  justification = centering,
  font = bf, % Bold table caption
  textfont = {bf}
}
\usepackage[table]{xcolor}
% Double spacing for the manuscript
\usepackage{setspace}
% \doublespacing

% Section Formatting
% \usepackage{titlesec}   
% \titleformat{\section}
%   {\large\bfseries}  % Large and bold
%   {\thesection\ \textbar}  % Section number followed by a vertical bar "|"
%   {0.5em}            % Space between "|" and title
%   {\MakeUppercase}   % Convert title to uppercase

\usepackage{xcolor}




\title{\textbf{Game structure obviousness in homegrown preference elicitation}}

\author[1]{Levenson Badio\thanks{Department of Agricultural Economics, Texas A\&M University, College Station, TX  77843 USA, tel:+1-9792407106 e-mail: \href{mailto:levenson.badio@tamu.edu}{levenson.badio@tamu.edu}.}}
\author[1]{Marco A. Palma\thanks{Professor and Director Human Behavior Laboratory, Department of Agricultural Economics, Texas A\&M University, College Station, TX  77843 USA, tel:+1-9798455284 e-mail: \href{mailto:mapalma@tamu.edu}{mapalma@tamu.edu}.}}
\author[2]{Andreas C. Drichoutis\thanks{Professor, Department of Agricultural Economics \& Rural Development, School of Applied Economics and Social Sciences, Agricultural University of Athens, Iera Odos 75, 11855, Greece, e-mail: \href{mailto:adrihout@aua.gr}{adrihout@aua.gr}.}}
\author[1]{Samuel Zapata\thanks{Associate Professor, Department of Agricultural Economics, Texas A\&M University, McAllen, Texas, USA, tel: +1-956-9685585 e-mail: \href{samuel.zapata@ag.tamu.edu}{samuel.zapata@ag.tamu.edu}.}}
\author[1]{Rodolfo M. Nayga, Jr.\thanks{Professor and Head, Department of Agricultural Economics, Texas A\&M University, College Station, TX  77843 USA, and Adjunct Professor, Korea University; tel:+1-9798452116 e-mail: \href{mailto:rnayga@tamu.edu}{rnayga@tamu.edu}.}}

\affil[1]{Texas A\&M University}
\affil[2]{Agricultural University of Athens} 
\date{}

\begin{document}


\maketitle
 \onehalfspacing

\begin{abstract}
\noindent The Becker-Degroot-Marschak (BDM) mechanism is widely used in experimental valuation research because of its theoretical incentive compatibility and implementation simplicity. However, recent work challenges the BDM's ability to match outcomes that closely represent its theoretical predictions in induced value (IV) settings. The Game Structure Obvious (GSO) mechanism is a new method that has shown promising results in aligning IV theoretical predictions with empirical findings. The GSO mechanism however has not been tested in homegrown settings, which are particularly important in environmental and market valuations for understanding demand for new or unfamiliar products. We are the first to test the BDM and the GSO mechanisms in an online homegrown experiment eliciting consumers' preference for oranges from trees treated with antibiotics, and compare the valuation outcomes in real and hypothetical treatments. This research question is important to identify the optimal tools for measuring food and resource valuations with incentives reflecting actual market values. Our findings reveal a counterintuitive methodological insight: while the GSO is perceived as simpler and more intuitive than the BDM mechanism in all treatments, the GSO mechanism induces significant hypothetical bias in early decision rounds, while the BDM mechanism does not. Valuations elicited in all other treatment are statistically indistinguishable. However, the GSO mechanism shows substantial learning effects, with hypothetical bias quickly diminishing over subsequent rounds. 

\textbf{Keywords:} Strategy-proof, BDM, GSO, hypothetical bias, citrus greening, oranges. 
	
\textbf{JEL codes:} C90, C91, D44, D80
 
 \textbf{AEA RCT Registry:} AEARCTR-0014656
 
 \end{abstract}


\onehalfspacing
\section{Introduction}
Obtaining accurate estimates of individual valuations for products and services is crucial for guiding policy recommendations and driving product development innovations. Experimental auctions and auction-like mechanisms are prominently used tools in valuation studies \citep{lusk2007experimental,canavari2019run}. However, reliably uncovering true preferences remains a persistent challenge--even in cases where the good in question has a simple, fixed, and objectively measurable value \citep{drichoutis2022game, cason_misconceptions_2014}. One commonly used method for eliciting preferences--especially in settings where direct interaction with participants is limited--is the Becker-DeGroot-Marschak (BDM) mechanism,  favored for its simplicity and theoretical incentive compatibility \citep{mamadehussene2023reliability, azrieli2018incentives}. The BDM mechanism is widely used in online studies to measure product valuations, beliefs, subjective probabilities, and trust  \citep{mamadehussene2023reliability, ahles_testing_2024, burdea2022online}. In this mechanism, participants submit a bid for a good, knowing that a price is randomly drawn from a predetermined range. If the participant's bid is equal to or exceeds the drawn price, they purchase the item at that price. Otherwise, they do not receive the item \citep{becker_measuring_1964}.

Despite its theoretical appeal, the BDM mechanism has shortcomings. %It  consistently underperforms relative to auction formats like the second-price auction \citep{DrichoutisEtAl2024incentives}. 
\citet{cason_misconceptions_2014} highlighted key pitfalls of the BDM mechanism within an induced value framework, where participants were asked to provide their willingness to accept (WTA) a card with a known fixed objective value of \$2. This study directly challenged the theoretical incentive compatibility of the BDM mechanism by showcasing that only 16.7\% of participants bid within 5\% of the actual value of \$2.\footnote{However, bidding accuracy (within the \$2) increases with learning and roughly doubles by the second round of bidding \cite{cason_misconceptions_2014}.} This issue is not unique to the BDM mechanism; \citet{DrichoutisEtAl2024incentives} report similar findings to \citet{cason_misconceptions_2014} with only 24.7\% of bidders submitting bids within 5\% of the induced value in the BDM mechanism, while second-price auctions show only modest improvements to 27.4\%. Given the consistent failure of participants to bid accurately even in simple, controlled settings with objectively defined values, the reliability of these mechanisms as tools for eliciting true market preferences warrants serious scrutiny.

The persistent failure of existing elicitation mechanisms to induce truthful bidding--even in settings where weakly dominant strategies should be obvious-- has motivated the development of new strategy-proof methods designed to make optimal choices more apparent to participants \citep{li_obviously_2017, pycia_theory_2023}. Strategy-proof mechanisms are those in which participants have no incentive to misrepresent their true preferences, as doing so offers no benefit. Among these, the Game Structure Obvious (GSO) mechanism has emerged as a novel and increasingly prominent tool. It is recognized both for being strategy-proof and for making weakly dominant strategies more transparent and intuitive to participants \citep{chakraborty_future_2025}. 

In the context of homegrown valuation, the GSO mechanism functions as a multiple price list in which participants progressively choose, at each step, whether to purchase or decline an item at a given price \citep{yu2021multiple, herberich2012digging, jack2022multiple}.  The price starts at a fixed level and increases by a constant increment until it reaches a terminal value randomly determined at the outset. We argue that comparing GSO with the BDM mechanism in a homegrown setting is essential for two reasons. First, in induced value settings, the GSO  has demonstrated significantly higher bidding rates closer to theoretical predictions--between 43\% and 58\%--compared to the BDM, which achieves a bid accuracy rate of around 15\% \citep{chakraborty_future_2025}. However, it is unclear whether these results transfer verbatim to a homegrown setting. Secondly, since the GSO mechanism has not yet been empirically applied in homegrown valuation studies, its reliability in settings involving hypothetical scenarios remains an open question. 
Unlike induced value studies, homegrown valuation involves items whose value is heterogeneous and not objectively defined. Consequently, we focus on which elicitation method produces valuations further away from purely hypothetical values.  This topic is important to address given that market and resource valuations are more complex and often involve the evaluation of multiple attributes and information.  Therefore, evaluating the performance of the GSO and BDM mechanisms under both hypothetical and incentivized conditions is critical to establish meaningful benchmarks for future use. 

To assess their performance in homegrown value settings, we implemented an online experiment recruiting participants representative of the U.S. population in order to test the performance of the GSO and BDM mechanisms under real and hypothetical incentives; the hypothetical treatments served as baseline. As a case study, we elicited consumers' willingness to pay (WTP) for fresh oranges from trees treated with trunk-injected antibiotics for Huanglongbing (HLB) and conventionally grown oranges. This application is relevant to the citrus industry that seeks to adopt economically and logistically feasible methods to control the disease while ensuring consumer acceptance. 
% This is too much information for the intro: Each participant received a \$6 endowment (except in hypothetical treatments, in which case participants were informed that decisions were not binding but encouraged to respond as if they were real). Elicitation tasks occurred over three rounds. Participants bid simultaneously on two types of oranges.

Our results point to a few stylized facts. First, subjects in our experiment find the GSO mechanism less complex than the BDM, consistent with its design goal of making optimal strategies more transparent. Paradoxically, while we find no evidence of hypothetical bias in the BDM mechanism, we find clear evidence of such bias in the GSO mechanism. However, when disaggregating the analysis by round, our results suggest that there is significant learning in the GSO mechanism such that hypothetical bias diminishes in subsequent rounds. In contrast, there is no evidence of learning in the BDM mechanism.
Furthermore, under real incentives, WTP elicited through the BDM and the GSO mechanism are statistically indistinguishable. However, under hypothetical incentives, the GSO mechanism consistently produces significantly higher WTP values than the BDM mechanism.

These findings provide a critical methodological contribution to the homegrown preference elicitation research. Contrary to expectations derived from induced value settings, this article reveals a nuanced and counterintuitive insight: although the GSO mechanism is perceived as more game obvious, it does not mitigate hypothetical bias. This challenges the common assumption that simpler and more transparent mechanisms inherently yield more accurate preference estimates \citep{li_obviously_2017}. At the same time, it highlights that results from induced value settings may not transfer verbatim to homegrown settings where preferences are subjective and context-dependent. The presence of hypothetical bias in the GSO is not due to mechanism understanding nor to selective attention. The GSO mechanism appears to require and facilitate learning processes more effectively over time, helping participants converge toward best-response behavior. In contrast, we do not observe such evidence of learning or improvement under the BDM mechanism.

We make several important contributions to the preference elicitation and experimental economics literature. First, we extend the growing literature on strategy-proof mechanisms by providing the first systematic evaluation of the GSO mechanism in homegrown valuation contexts, where true values are subjective and unknown \citep{li_obviously_2017, pycia_theory_2023, chakraborty_future_2025}. While previous research has demonstrated GSO's superiority in induced value settings with known objective values, our findings reveal that these advantages do not transfer directly to real-world valuation scenarios involving consumer preferences. Second, our work contributes to the hypothetical bias literature \citep{penn2018understanding, cummings1999unbiased, loomis_whats_2011, fang_use_2021, list2001explicit, grebitus2013explaining} where we uncover a counterintuitive finding: the GSO mechanism, despite being game strategy obvious and superior to the BDM mechanism in IV settings, exhibits significant hypothetical bias while the BDM mechanism does not (at least not in this study). This suggests that the relationship between mechanism complexity and bias mitigation is non-monotonic and context-dependent. Third, we advance understanding of learning effects in preference elicitation \citep{drichoutis2011role, canavari2019run} by demonstrating that the GSO mechanism facilitates dynamic learning processes that reduce hypothetical bias over successive rounds, while the BDM mechanism shows no such learning patterns. This finding has important implications for multi-round experimental designs and suggests that GSO may be particularly valuable in educational or repeated-interaction contexts. Finally, our study provides methodological guidance for applied consumer research by showing that elicitation method choice can substantially affect estimated valuations, particularly under hypothetical conditions, reinforcing the importance of careful mechanism selection in valuation studies\citep{miller2011should, schmidt2020accurately}.

Our paper proceeds as follows. Section~\ref{Experiment} describes the experimental design, procedures and data collection, section~\ref{Econometric} presents the empirical analytical methods, section~\ref{Results} shows the results, and close with the section ~\ref{Conclusion}.

\section{Experimental methods}
\label{Experiment}
In the next subsections, we describe the experimental protocol that we followed, a description of the methods, procedures and data collected to address our main research question. 

\subsection{Experimental Design}
Our study employed a $2\times2$ between-subjects design that varied across two dimensions: elicitation method (BDM vs. GSO mechanism) and incentives (Hypothetical vs. Real incentives). Subjects were randomly assigned to one of the four treatment groups. In the hypothetical incentives treatment, subjects were explicitly informed that none of the described payoffs or potential purchases were real or were actually going to take place and they would only receive a participation fee. In the real incentives treatment, we implemented a Between-subjects Random Incentivized Scheme (BRIS) with a 10\% realization probability \citep{ahles_testing_2024}. On top to their participation fee, each participant was endowed with \$6 that they could potentially use to buy a 3.2 pound bags of oranges during the experiment with the understanding that any money they did not use would be added to their bonus payment. Subjects submitted bids in three consecutive rounds, and at each round they could bid for a bag of conventional oranges and a bag of oranges from trees treated with antibiotics. Participants were aware that at the end of the experiment one round and one product would be randomly drawn as binding and any transactions regarding that product and round would be realized.\footnote{ However, for the hypothetical treatments, participants were explicitly informed of the hypothetical nature of the endowment and decisions.}  

The elicitation task was carried out in three rounds to assess the role of learning \citep{corrigan2008testing, drichoutis2011role} in value elicitation. The first two rounds of bidding were exactly the same (i.e., no new information was added between the rounds) and with only basic information about the types of oranges as normally presented in real markets. Before the third round, we introduced an information treatment video that informed subjects about the impact of HLB on the US citrus industry.\footnote{Participants watched a 51 seconds \href{https://www.youtube.com/watch?v=_AqMBjB0ChM}{video} containing objective information about the impact of HLB on the US citrus industry, the livelihood of farmers, and its potential risk to human health and the environment. The information script was revised by agricultural scientists to ensure the accuracy and veracity of the information. This design allows us to measure the impact of information provision on WTP for oranges from antibiotic-treated orange trees relative to conventional oranges.} In both the BDM and the GSO mechanism treatments, the prices were randomly drawn from a uniform distribution of [\$0, \$6] with increment of \$0.20.\footnote{We chose the 20 cents increments to balance between bid precision, time and cognitive load, following the literature using an increment representing 3.33\% of the endowment leading to a maximum of 31 decisions per round \citep{li_obviously_2017, chakraborty_future_2025}. The upper bound of the range was chosen based on the actual maximum value of the orange prices in the market at the time of the study plus a price premium of 40\% higher than the market price for those willing to pay a premium for oranges from antibiotic-treated orange trees.}

In the BDM mechanism treatment, subjects had to use a slider to select their bid. If their bid for 3.2 pound bag of oranges was greater than or equal to the randomly drawn price, they would buy the product at the randomly drawn price, otherwise, they would not buy the product and would keep their endowment.

In the GSO treatments, at the start of each round, an `Offer Price' appeared on the screen that was subsequently increased in fixed increments of \$0.20 after each decision they made. The offer price for the first screen was set to \$0. The process of increasing the offer price continued until it reached a randomly drawn `market price', which was unknown to participants. The random price was independently drawn for each round from a uniform distribution between \$0 and \$6 and was different between subjects, rounds and treatments. When the offer price matched the randomly drawn price, the round automatically ended.\footnote{See Figure~\ref{fig:Appendix_GSO_game} in the Appendix for an illustration and a link to a video capture of how this dynamic process looked like.} 



One characteristic of the GSO mechanism is that it does not prevent multiple switching behavior (MSB). Specifically, individuals who may initially decide not to purchase at a given (low) price may attempt to buy at a higher price in subsequent rounds. To reduce MSB due to accidental click or lack of attention, we implemented a soft nudge consisting of a warning to MSB, reminding them of the last price they chose not to buy, with the option of changing their inconsistent choice \citep{yu2021multiple}.\footnote{Participants retained full freedom to make choices as they wished. To address potential input errors (e.g., accidental clicks), a confirmation prompt was displayed whenever a participant switched from opting to buy to not buying: ``You are deciding not to buy at this price. Do you want to confirm it?''} 
This intervention was designed to reduce MSB caused by inattention. Our main analysis does not exclude subjects that exhibited MSB in the GSO treatment, as these participants may genuinely prefer randomization \citep{agranov2017stochastic}. However, we report results excluding these participants as a robustness check.
    
\subsection{Experimental Procedures}
The experiment was programmed in Qualtrics and the study protocol was approved by the Institutional Review Board (IRB) at \hl{blind if submitting to AJAE} Texas A\&M University (approval number: STUDY2024-1002). Our study was pre-registered with the AEA's RCT registry (\href{https://www.socialscienceregistry.org/trials/14656}{AEARCTR-0014656
}). A large representative sample of 2,323 participants were recruited through Forthright Access--an online research company that handles their own recruitment through a variety of direct advertising channels-- and initially started the study. Of these, 1,086 subjects either declined, failed to meet the screening criteria or the attention checks. This resulted in a final sample of 1,237 participants with complete responses. The data were collected in December 2024. To be eligible, participants had to be at least 18 years old and purchase and eat citrus at least once a month. 
Each participant received a base participation fee of \$2 for completing the study. The study started with participants receiving treatment-specific instructions, followed by comprehension questions. In the real incentivized treatment, participants were informed that they would complete three rounds of elicitation tasks and that one product from a randomly chosen round would be randomly chosen for realization. 

After completing the main bidding task, participants responded to a post-experiment survey to measure perceived complexity and choice certainty.\footnote{We measure perceived complexity by asking how complex they found the valuation task and choice certainty by asking how frequently they felt unsure or inclined to revise their valuations.} For the real incentivized treatments, a random number determined the realization with 10\% probability.  If selected, the winner would receive any remaining funds from their endowment and was asked to voluntarily provide a shipping address to deliver (free of charge) the oranges via priority mail. A separate link was provided to enter the mailing address information to preserve anonymity. A total of 108 (8.7\%) participants were selected for payment, and we shipped oranges to 27 subjects. The rest were non-buyers who just received their endowment of \$6 on top of their participation fee. 
%Among the 45 buyers, 36 provided a mailing address and were subsequently shipped the oranges; they also received any remaining balance from their endowment.

% \subsection{Data}


\section{Econometric analysis}
\label{Econometric}
Having established our experimental protocol, we now turn to the econometric analysis that will test our hypotheses about relative mechanism performance. To analyze the data, we use interval regression models to account for the nature of the interval data elicited with the GSO mechanism as well as the panel nature of the data. Unlike standard valuation methods, the GSO mechanism captures WTP within an interval, rather than a point value. Specifically, a participant's valuation is bounded between the switch point between `Try to Buy' and `Do Not Buy'. The model can also accommodate point valuations elicited with the BDM mechanism. In our data, 1.5\% of observations is left censored at \$0, while 3\% is right censored at \$6. To improve inference under potential heteroskedasticity, we applied 1,000 bootstrap iterations to estimate robust standard errors.

To compare elicited valuations between the treatments, we use the following specification:
\vspace{-1cm}

\begin{equation}\label{eq:specification}
Y_{it}^* = \gamma_0 + \gamma_1 \cdot \text{GSO}_i + \gamma_2 \cdot \text{Hypo}_i + \gamma_3 \cdot \text{(GSO} \times \text{Hypo)}_i +  \varepsilon_{it} 
\quad \varepsilon_{it} \sim \mathcal{N}(0, \sigma^2)
\end{equation}


In Equation~\ref{eq:specification}, $Y_{it}^*$ denotes the unobserved WTP for individual $i$ at round $t$, with $t = 1, 2, 3$. For the BDM mechanism treatment groups, it is  $Y_{it}^*$= $Y_{it}$ i.e., we observe point data; for the GSO mechanism, we only observe WTP in intervals defined by subjects' switching point. In Equation~\ref{eq:specification}, $\text{GSO}_i$ is a dummy variable indicating assignment to the GSO mechanism treatment and $\text{Hypo}_i$ is a dummy variable indicating assignment to the Hypothetical incentives treatment. $\varepsilon_{it} \sim \mathcal{N}(0, \sigma^2)$ is an idiosyncratic error term for individual $i$ at round $t$.

Alternative specifications to Equation~\ref{eq:specification}, control for rounds (third round also coincides with information provision), type of oranges and MSB.

\section{Results}
\label{Results}
%\hl{Levenson, I think the alternating grey color for rows for all Tables will be disliked by journals. Please go for simple tables using the toprule, midrule and bottomrule lines to separate sections in tables. horizontal lines should be used at a minimum and completely avoid vertical lines.}

Given that 2,295 participants started the experiment but only 1,184 completed the study, we first examine whether the final sample differs systematically in terms of observable characteristics. To do so, we estimate a logistic regression in which the dependent variable is a binary indicator that is equal to one if participant failed to complete the survey, and the explanatory variables are demographic characteristics. Our results suggest that only age and high income differ significantly. Other variables such as race, gender, education, region, education, marital status do not affect participants not completing the experiment (see Table~\ref{tab:Incomplete} in Appendix).

To check whether randomization worked as expected, we use standardized differences across treatments \citep{CochranRubin1973}. We find that our sample is well balanced.\footnote{While many researchers use statistical tests to check for balance of observable characteristics between treatments, the literature points to some pitfalls of this practice \citep[e.g.,][]{canavari2019run,DeatonCartwright2016,BrizEtAl2017,HoEtAl2007,MoherEtAl2010,MutzPemantle2015}. Following this literature, we report in Table \ref{tab:Appendix_std_diff_table} in the Appendix standardized differences across treatments \citep{ImbensRubin2016,ImbensWooldridge2009}. \citeapos{CochranRubin1973} rule of thumb is that the standardized difference should be less than 0.25. }


The main results are based on data that include MSB, as this might reflect indecisiveness in preferences elicited with the GSO mechanism. Moreover, since the BDM includes all MSB types (because subjects could not engage in multiple switching), a fair comparison to the BDM is the full sample in the GSO which includes all MSB types. As a robustness check, we also report results excluding MSB. Doing so, does not affect our main findings. Following the approach of \citet{brown2018separated}, we included these participants in the main analysis, interpreting MSB as a potential indication of a preference for randomization within an indifference range \citep{agranov2023stable}. 

Additionally, given that MSB is prevalent in a multiple price list format--ranging from 8.5\%  to up to 50\% in the literature \citep{yu2021multiple, filippin2016reconsideration}-- we implemented a simple nudge aimed solely at reducing inattentive behavior and accidental clicks. This intervention led to a substantial improvement in participant behavior: the MSB rate declined from 14.31\% in round 1, to 9.60\% in round 2, and further to just 3.87\% in round 3.

By comparison, \citet{yu2021multiple} report that their nudge reduced MSB by about 26\%. Our intervention not only lowered MSB rates significantly but also appeared to promote learning, with the MSB rate stabilizing at around 4\% by the third round. These results suggest that our nudge was effective in helping participants better understand and apply their best-response strategies. 

We estimated interval regression models given the specification in Equation~\ref{eq:specification}, with and without demographic controls. We then calculated marginal effects
Our main regression analysis results are summarized in Table~\ref{tab: Regression}. 



\begin{table}[htbp]
\centering
\footnotesize
\caption{Marginal effects from RE interval regression models}
\label{tab: Regression}
\begin{tabular}{llcccc}
\toprule
 & & \multicolumn{2}{c}{ \makecell{Model (1) \\ (w/o demographics)}} & \multicolumn{2}{c}{\makecell{Model (2) \\ (w/ demographics)}} \\
 & & Marginal effect & Std. Error  & Marginal effect & Std. Error \\ \midrule
\multirow{2}{*}{$\frac{\partial Y^*}{\partial GSO}$} & $Hypo=0$ &0.127 &0.067 &0.09 & 0.076\\
                                               & $Hypo=1$ & 0.352***&0.077 & 0.39*** & 0.069\\ \midrule
\multirow{2}{*}{$\frac{\partial Y^*}{\partial Hypo}$} & $GSO=0$ &-0.025 &0.061 &-0.037 &0.061 \\
                                                & $GSO=1$ &0.198**  & 0.083&0.26*** & 0.084\\ \bottomrule


\end{tabular}
\begin{tablenotes}
\footnotesize
\item Notes: This table shows the marginal effect comparing elicitation methods and incentives.
\item Standard Error are bootstrapped 1000 times
\end{tablenotes}
\end{table}



\textbf{Result 1:} BDM Mechanism Yields Similar WTP in Real and Hypothetical Settings 

 The main objective of this research paper is to compare the BDM and GSO performance in homegrown settings in terms of hypothetical bias mitigation. We first investigate whether BDM leads to hypothetical bias. Referring to Table~\ref{tab: Regression}, we can see that the hypothetical bias value is close to zero and is not statistically significant. This result remains consistent after controlling for other variables. A graphical representation of this result is shown in Figure~\ref{fig:margin_hypo}. This finding aligns with a growing body of experimental literature suggesting that, under certain conditions, individuals can provide valuations in hypothetical settings that closely approximate those elicited under real monetary incentives.\citep{branas-garza_paid_2023, drichoutis_incentives_2025}. 
 
 %\hl{Most other papers show hypothetical bias so not sure what is the right balance to show evidence against the majority of previous studies.}
 
\vspace{0.5cm}

\textbf{Result 2:} GSO Mechanism Yields statistically different WTP in Real and Hypothetical Settings 

Turning our attention to GSO under different incentives scheme, our results provide clear evidence of hypothetical bias: valuations in the hypothetical setting are statistically higher than those in the non-hypothetical setting (\(\Delta = 0.198\), \(p =0.016\)). This represents  an increase of  5.5\% (see column 3 in the regression Table \ref{tab:interval_regression}). This finding aligns with previous literature suggesting that people tend to overstate their valuations in hypothetical scenarios compared to real financial settings \citep{penn2018understanding, fang_use_2021}. Figure~\ref{fig:margin_hypo} graphically illustrates this difference, showing that GSO valuations under the hypothetical setting are significantly higher at the 95\% confidence level. When applying Bonferonni correction, This result remains significant at a p-value of 10\%, if not controlling for other variables, adding control, it is significant \((p = 0.01)\).


This result is particularly noteworthy—and somewhat surprising—because we observe hypothetical bias in the GSO mechanism but not in the BDM. This is unexpected, as GSO is generally considered more game-transparent than BDM and is often thought to elicit valuations that more closely reflect theoretical predictions in induced value settings. We explore possible explanations for this outcome in Section 5.3.

\vspace{0.5cm}



\textbf{Result 3:} The GSO and BDM yield statistically equivalent estimates when using real financial incentives. 

We now compare GSO and BDM under incentivized conditions. Given that incentives represent a golden rule in exprimental economics \citep{smith_experimental_1976}, testing these two mechanisms under such conditions is important to evaluate whether they deviate from each other, or whether GSO and BDM valuation under such condition. Table \ref{tab: Regression} supports that under the real GSO provides valuation that are are not statistically different, they differ in magnitude by about \$0.13 (\(\Delta = 0.127\), \(p = 0.101\)). When controlling for other variables (demographics, round, risk preference), GSO and BDM, they remain statistically equivalent (\(\Delta = 0.09\), \(p = 0.22\)). This result suggests that under real incentives, participants in the GSO and BDM elicit valuations estimates that are statistically equivalent.

\vspace{0.5cm}

\textbf{Result 4:} Under hypothetical scenario, GSO produces valuation that are statistically higher than BDM.

One thing to note in Figure \ref{fig:margin_hypo}, is that GSO provides statistically higher valuation also in magnitude as compared to the other treatments. This supports the results previously discussed. Specifically, when comparing valuations in the hypothetical setting, we find significantly higher values under GSO compared to BDM  (\(\Delta = 0.35\), \(p = 0.000\)). GSO  is 10\% higher as compared to BDM in hypothetical setting. Given that we previously detected hypothetical bias in GSO but not in BDM, this raises concerns about the reliability of GSO-based valuations under hypothetical conditions. Based on these findings, BDM appears to be a more reliable elicitation method than GSO for hypothetical experimental settings.

\vspace{0.5cm}

%\textbf{Result 5:} Hypothetical bias is statistically higher in GSO compared to BDM.

%Our main research question was to study which elicitation method entails a higher hypothetical bias. However, to better assess its reliability, we needed to compare them under different incentive structures. 
%Since GSO and BDM  are equivalent under real but differ under hypothetical settings, we test them under different incentive structures. Our result suggests that hypothetical bias in GSO is statistically higher than BDM without any control variables or correcting for Bonferonni (\(\Delta = 0.22\), \(p = 0.029\)). Applying Bonferonni correction, it becomes insignificant. However, when adding controls and correcting for Bonferonni, the result shows higher level of hypothetical bias in the GSO as compared to the BDM (\(\Delta = 0.29\), \(p = 0.02\)).

%\hl{What exactly is Result 5 giving us that warrants a separate result? Could we embed this in another result above?}




            
\subsection{Investigating the Source Mechanism for hypothetical bias}

 Given that our results indicate the presence of hypothetical bias under the GSO mechanism, we proceed to analyze the underlying factors driving this bias. Specifically, we examine four potential mechanisms that may explain such bias: (1) mechanism comprehension differences,(2) social desirability bias \citep{lopez2021social, norwood2011social,bursztyn2025social}, (3) familiarity with products \citep{fox1998cvm, veettil_hypothetical_2024}   (4) Multiple Switching \hl{need a rationale for how MS can result in hypo bias}, (5) learning and (5) attention allocation. Each represents a different pathway through which hypothetical bias may be moderated.
 



\subsubsection{Participants' understanding of the mechanism and game form recognition}We first measure whether GSO leads to higher game form recognition--defined as participants’ understanding of the mapping between their strategy profile (or choice) and the resulting outcomes--  as compared to BDM. We conducted a chi-square test to compare proportion of participants who find that the elicitation mechanism is complex in BDM and GSO. The result also confirms that a higher proportion of people in the BDM report high difficulty understand it (see Table~\ref{tab:Appendix_proportion_test} in Appendix).
We first measure whether GSO leads to higher game form recognition--defined as participants’ understanding of the mapping between their strategy profile (or choice) and the resulting outcomes--  as compared to BDM. We conducted a chi-square test to compare proportion of participants who find that the elicitation mechanism is complex in BDM and GSO. The result also confirms that a higher proportion of people in the BDM report high difficulty understand it (see Table~\ref{tab:Appendix_proportion_test} in Appendix).
We now examine whether misconceptions contribute to hypothetical bias in the GSO, we measure whether participants' understanding of the elicitation is a predictor of hypothetical bias. To do so, we use their overall test score in the GSO comprehension assessment task and interact it with hypothetical bias (see table \ref{tab:MSB_interval_regression_Score_overall} in appendix). The results do not suggest that a lack of understanding of the mechanism is a driver of the hypothetical bias observed in the GSO (\(\Delta = 0.062\), \(p > 0.10\)). In addition, we analyze responses from participants in the top 25\% (highest quartile) and bottom 25\% (lowest quartile) based on their assessment scores (see Table~\ref{tab:MSB_interval_regression_BDM_Scorehigh_low}).  We find that hypothetical bias is actually greater among participants with higher comprehension. \hl{This needs to be tried to be explained. Can you try to find some literature around this to explain? Is it possible that does who understand correctly engage in other motivations such as social desirability bias? Could you look at the correlation?}

Finally, as shown in table \ref{tab:Appendix_gameform} in appendix,  we analyze participants perceived complexity of the mechanism, the result suggests that participants in the BDM report a higher level perceived complexity as compared to GSO. Additionally, we measured choice certainty, as how often participants feel like changing their bids. Using this metric, we find GSO leads to a higher choice certainty as compared to BDM. These two results suggests that GSO is more game obvious than BDM as supported by induced value experiment.  \citep{chakraborty_future_2025,brown_is_2023}.
Overall, these results suggest that lack of understanding is not a threat in our study nor is a driver of hypothetical bias detected in the GSO.

\subsubsection{Social desirability}
Social desirability bias is a known contributor to hypothetical bias in stated preference studies \citep{norwood2011social, entem2022using}. Given the balanced distribution of observable characteristics across treatments, and the absence of hypothetical bias in the BDM condition, we would not expect social desirability to be the driver in the GSO, we would expect it to be constant. Nevertheless, we examine its potential influence on both mechanisms within our experimental framework. Specifically, we focus on participants who express favorable views toward antibiotic-treated citrus (specifically those who confirms they would support the use of antibiotics in oranges treatment).  We compare hypothetical bias of these participants before using the information as compared to after receiving information. This allows us whether this leads to valuation inflation after being treated with information. As shown in Table \ref{tab:interval_regression_socialdesirability_BDM} (left-hand side), the results do not indicate that participants with positive attitudes toward genetically modified (GM) products systematically inflate their valuations after receiving that information as compared to before receiving information in the GSO (\(\Delta = -0.199\), \(p > 0.10\)). \footnote{Neither have we found such evidence in the BDM mechanism (\(\Delta = -0.158\), \(p > 0.10\)).} This suggests that social desirability does not drive hypothetical bias in the GSO.

\subsubsection{Familiarity with products}
We assess whether familiarity with the product influences hypothetical bias in the GSO mechanism by comparing WTP for two types of orange products: one conventional and highly familiar to consumers, and the other from trees treated with antibiotics injection. Similar to \citet{veettil_hypothetical_2024},  we found that product familiarity does not significantly moderate hypothetical bias in neither the GSO (\(\Delta = -0.01\), \(p > 0.10\)), nor in the BDM (\(\Delta = 0.045\), \(p >0.1\)). We can omit this as potential factor moderating the hypothetical bias (see Table~\ref{tab:Orange_socialdesirability_BDM} in the Appendix.)




\subsubsection{Analysis by Multiple Switching}
We first compare participants with and without MS to investigate whether it reflects a preference for randomization (e.g., indecisiveness) or confusion about the mechanism. To explore this, we examine participants' performance and their perceived complexity within the GSO elicitation method, comparing non-MS to MS.

A t-test shows no significant differences in test scores between non-switchers and MS in their assessment test score, suggesting that overall comprehension does not differ meaningfully across these groups. To complement this, we used a chi-square test (see Table \ref{tab:Appendix_proportion_test_groupMSB_vs_Non-MSB}), we find that a significantly higher proportion of MSB participants reported perceiving the mechanism as highly complex compared to non-MSB participants (\(\Delta = 14\%\), \(p < 0.001\)). This suggests that MS may be more a result of confusion rather than of a deliberate preference for randomization.


Finally, we assess whether hypothetical bias is higher for MS as compared to non-MS, we create a dummy variable indicating MSB and interact it with the incentive condition. The results (Table \ref{tab:MSB_interval_regression_MSB}, Appendix) indicate that MSB does not significantly influence hypothetical bias (\(\Delta = 0.107\), \(p > 0.10\)). We further validate this finding using three model specifications: (1) including all participants, (2) excluding MS, and (3) analyzing only MS (see Table \ref{tab:MSB_interval_regression_multiple_comaparisons} in the Appendix). While results indicate valuation inflation among MS under both real and hypothetical scenarios, we find no statistical evidence that MS drives the hypothetical bias observed in the GSO mechanism. Therefore, MS can be ruled out as the primary source of hypothetical bias in this study.

In summary, MS does not appear to reflect a fundamental misunderstanding of the mechanism as also found by \citet{yu2021multiple}, nor does it exacerbate hypothetical bias. Rather, it may be associated with valuation inflation, possibly indicating lower-quality responses among those participants.








\subsubsection{Learning in GSO decreases hypothetical bias}
As shown, in Table \ref{tab: Regression by Round} , We analyze hypothetical bias round-by-round to see whether learning can contribute to mitigating hypothetical bias gap. For BDM, there is no evidence of learning leading to changes of bidding behavior, the bids are relatively stable across the rounds (see Figure \ref{fig:Learning} in Appendix).
For GSO, focusing on Round 1, we observe that the hypothetical bias is the highest (\(\Delta = 0.31\), \(p < 0.01\)). In Round 2, there is evidence of learning, as the hypothetical bias decreases by 55\% (\(\Delta = 0.238\), \(p = 0.001\)). In Round 3, there is no evidence of hypothetical bias in GSO (\(\Delta = 0.21\), \(p = 0.007\)). Consistent with \citet{brown_is_2023}. Using the Bonferonni correction, the results of round 2 and 3 become significant at a p-value equal to 10\%. Overall, we find that that participants learn to play their weakly dominant strategy over the rounds, and this significantly lead to hypothetical bias over the rounds. This learning process not only reduced the magnitude of the hypothetical bias gap in subsequent rounds but also diminished its statistical significance. One of our limitations, we could not cleanly extend and see participants learning pattern beyond round 2 and 3. In future applications of GSO for preference elicitation, it would be beneficial to increase the number of rounds to measure this learning pattern.

\begin{table}[htbp]
\centering
\footnotesize
\caption{Marginal effects from RE interval regression models}
\label{tab: Regression by Round}
\begin{tabular}{llccc}
\toprule
 & & Round number & \multicolumn{2}{c}{\makecell{Model (1) \\ (w/ demographics)}} \\
 & &  & Marginal effect & Std. Error \\
 \midrule
\multirow{2}{*}{$\frac{\partial Y^*}{\partial Hypo}$} & $GSO=0$ & 1 & -0.05 & 0.071 \\
                                                     & $GSO=1$ & 1 & 0.326*** & 0.10 \\ 
\midrule
\multirow{2}{*}{$\frac{\partial Y^*}{\partial Hypo}$} & $GSO=0$ & 2 & -0.08 & 0.07 \\
                                                      & $GSO=1$ & 2 & 0.237** & 0.10 \\ 
\midrule
\multirow{2}{*}{$\frac{\partial Y^*}{\partial Hypo}$} & $GSO=0$ & 3 & 0.02 & 0.061 \\
                                                      & $GSO=1$ & 3 & 0.21** & 0.10 \\ 
\bottomrule
\end{tabular}
\begin{tablenotes}
\footnotesize
\item Notes: This table shows the marginal effect comparing hypothetical bias in GSO and BDM in each round.
\item Standard Errors are bootstrapped 1000 times.
\end{tablenotes}
\end{table}



\subsubsection{Attention in the tasks}

Following \citet{chabris2008measuring}, we measure response deliberation time in elicitation tasks to assess whether this hypothetical bias in the GSO is driven by that. Specifically, we use the time per increment that we obtain by standardizing the number of decisions faced by each participant, we divide the final price at which the game ends by 0.2. This gives a consistent measure of the number of increments each participant encountered, since each increment corresponds to an increase of \$0.20 increase. In other words, when the random price is \$X, the participant faced 5X increments decisions. Given that participants are not paid in the hypothetical treatments, if this was the driver of hypothetical bias, if lack of attention was a driver of hypothetical bias, we would expect them to spend significantly less time in the hypothetical treatment.
Doing a t-test, In the GSO, we found that participants in the hypothetical treatment group allocate on average significantly less time as compared to those in the hypothetical one (\(\Delta = -1,964\), \(p = 0.039\)) (see table \ref{tab:my_label} in the Appendix). That is to say to say participants in the real GSO spend on average 1.5 times the amount of time than those in the hypothetical one. However, in the BDM, we find no statistical different between real and hypothetical treatments. Hence, we can omit selective attention as driver of hypothetical bias found in the GSO. 

Since we found that the magnitude and statistical significance of hypothetical bias decreased over the rounds, we further investigated whether this trend is related to participants’ attention. Specifically, we measured hypothetical bias across the rounds within the GSO treatment. The results reveal an interesting behavioral pattern. On average, participants in both incentive schemes spent significantly more time in Round 1 compared to Rounds 2 and 3 during the elicitation task (see the second and third rows of Table~\ref{tab: Marginal Effects by Round}). This suggests that participants were likely using the first round to learn the mechanism.

Furthermore, we found that in Round 1, participants in the Real incentive condition spent substantially more time—approximately 2.9 times longer—than those in the Hypothetical condition (\(\Delta = -4,967.85\), \(p = 0.098\)). In Rounds 2 and 3, this time difference decreased both in magnitude and in statistical significance, likely reflecting that learning had taken place across conditions. Given that we control for demographics as well, we also found that people with higher education tend to spend on average more time than people with no bachelor degree, and also older people tend to spend more time than their younger counterparts (see Table~\ref{tab:Appendix time}). 

Taken together, these findings indicate that the hypothetical bias observed in the GSO is correlated with differences in time spent between the two incentive schemes. In particular, participants in the Real treatment appear to invest significantly more attention early on, which may contribute to the initially stronger hypothetical bias.

\begin{table}[H]
\centering
\footnotesize
\caption{Marginal effects of Hypothetical Incentive by Round}
\label{tab: Marginal Effects by Round}
\begin{tabular}{llcc}
\toprule
 & Description & Marginal effect & Std. Error \\
 \midrule
\multirow{3}{*}{$\frac{\partial Y^*}{\partial Hypo}$} 
    & Round 2 \& 3 = 0 & -4967.85* & 3005.79 \\
    & Round 2 = 1 \& Round 3 = 0 & -732.19 & 494.13 \\
    & Round 3 = 1 \& Round 2 = 0 & -674.93 & 565.45 \\
\midrule
\multirow{2}{*}{$\frac{\partial Y^*}{\partial Round2}$} 
    & Hypo = 1 & -1054.09*** & 400.87 \\
    & Hypo = 0 & -5289.76* & 2956.29 \\
\midrule
\multirow{2}{*}{$\frac{\partial Y^*}{\partial Round3}$} 
    & Hypo = 1 & -1224.73*** & 399.97 \\
    & Hypo = 0 & -5517.65* & 2958.74 \\
\bottomrule
\end{tabular}
\begin{tablenotes}
\footnotesize
\item Notes: Marginal effects are derived from a linear regression model predicting the continuous outcome across rounds by incentive type.
\item The reported effects represent the discrete change from the base category (Real incentive).
\item Standard errors are robust to heteroskedasticity.
\end{tablenotes}
\end{table}




%\subsubsection{Robustness analysis}

%As a robustness check, we run similar analysis as the previous one in the result section by excluding MS. The results are robust to the exclusion of MS. (see Table \ref{tab:MSB_interval_regression_multiple_comaparisons} in appendix). 
 




\section{Conclusion}
\label{Conclusion}

In this study we address the fundamental question of how GSO, an elicitation known to be game obvious, performs as compared to BDM (a largely used preference elicitation method) in real consumer valuation contexts in hypothetical bias mitigation. The experiment involved a randomized online survey, participants were assigned to one of four treatment groups: GSO with real incentives, GSO hypothetical, BDM with real incentives, and BDM hypothetical. Our results show that, as predicted by theory, GSO is more game-obvious than BDM—that is, participants more easily understood the incentives involved. However, BDM outperformed GSO in mitigating hypothetical bias. Interestingly, the two mechanisms perform equivalently when real incentives were provided. Moreover, round-by-round analysis shows that participants in the GSO treatment exhibit a decreasing hypothetical bias over time, whereas bids in the BDM group remain relatively stable across rounds. This suggests that GSO’s transparent structure supports strategic learning, enabling participants to gradually converge toward truthful valuation--a valuable feature for multi-round experimental designs. 
%While BDM may be a more effective mechanism when participants have higher levels of formal education, this advantage may not hold in contexts where individuals often lack such education or familiarity with economic mechanisms like randomization procedures, such as in many developing countries\citep{burchardi2021testing, akbarpour2020credible}. 
In context, when multiple rounds is feasible, GSO may be more appropriate, since it is perceived to be less complex and facilitates learning over time, as demonstrated in this study. 


Our findings have important methodological implications for mechanism design and experimental economics. First, the relationship between a mechanism’s obviousness and its ability to elicit truthful valuations is non-monotonic. While simplification can enhance comprehension such as in the case of GSO, excessive simplicity, particularly in the absence of real incentives may trigger heuristic decision-making, leading to noise rather than accuracy. This highlights the need for careful design calibration, balancing clarity with enough structural complexity to maintain participant engagement and reflective responses. Second, the GSO mechanism may be particularly well-suited to experimental contexts that involve multi-stage bidding. Its intuitive structure appears to support learning, allowing participants to gradually identify their best responses over successive rounds. In such settings, GSO offers a promising approach for promoting strategy-proof behavior, especially when participants are unfamiliar with more abstract elicitation formats like BDM. Overall under hypothetical scenario, our result suggests BDM is more reliable. However under real setting, especially when there is an opportunity to learn the mechanism more through repeated rounds, the GSO would be recommended more than the BDM. 

One limitation of the study that merits consideration is that it is unclear whether GSO's learning advantages compound over many rounds or plateau after initial strategy discovery. In our case, since information was introduced before the third round, it may interact with learning.
Two promising avenues emerge from this work. Examining these mechanisms' performance across different contexts (e.g., different product categories, different stakes, environments) would test the robustness of our findings. Additionally, future research should examine whether the observed learning behavior in the GSO, but not in the BDM, is unique to these two elicitation methods—or whether multi-step games inherently promote greater learning than one-shot games in multi-round preference elicitation.


\bibliography{Refer_Antibiotics}


\newpage
	\singlespacing
	\appendix
	\setcounter{table}{0}
	\setcounter{figure}{0}
	\renewcommand{\thetable}{A\arabic{table}}
	\renewcommand{\thefigure}{A\arabic{figure}}
	\setcounter{page}{1}
	\renewcommand{\thesubsection}{\Alph{subsection}}

	\section*{\centering{Electronic Supplementary Material of}}
	{\centering \LARGE Game structure obviousness in homegrown preference elicitation
		
		\vspace{0.5cm}
		\renewcommand*{\thefootnote}{\fnsymbol{footnote}}
		\setcounter{footnote}{0}
		
		\large
        }


        
\section{Research protocol}

Welcome to a study of how people make economic decisions!Please read the instructions carefully and click the continue button if you agree with the Informed Consent below.

\textbf{Title of Research Study}: Consumer economic decision-making

\textbf{Investigator}: Dr. Samuel Zapata

\textbf{Why am I being asked to take part in this research study?}

You are invited to participate in this study because we are trying to learn more about consumers' preferences for beef. You were selected as a possible participant in this study because you are a U.S. consumer. You must be 18 years of age or older to participate. \par
\textbf{Why is this research being done?} \par

The survey is designed to understand consumer economic decision making. \par

\textbf{How long will the research last?} \par
It will take about 15 minutes. \par
\textbf{What happens if I say "Yes, I want to be in this research"?} \par
If you decide to participate, you will be asked to provide your valuations for several products and answer a survey. \par


\textbf{
What happens if I do not want to be in this research?} \par
Your participation in this study is voluntary. You can decide not to participate in this research and it will not be held against you. You can leave the study at any time. \par

\textbf{Is there any way being in this study could harm me?}
There are no sensitive questions in this survey that should cause discomfort. \par

\textbf{What happens to the information collected for the research?} \par
You may view the survey host’s confidentiality policy at: \href{https://www.beforthright.com/privacy}{Forthright Privacy Policy}. Your email address or other contact information will be stored separately from your survey data, and is only being collected for those who purchase products so we can mail them their products to their designated mailing address. All identifiable information will be kept on a password protected computer and is only accessible by the research team. Compliance offices at Texas A\&M may be given access to the studies files upon request. Your information will be kept confidential and allowed by law. The results of the research study may be published but your identity will remain confidential. \par

\textbf{Will my information be used for other research?} \par
Information collected as part of the research will NOT be used for other studies or given to other researchers for future studies. \par

\textbf{What else do I need to know?} \par

If you agree to take part in this research study, we will provide you with  \$3 compensation and may earn additional compensation in cash or a food product that, if eligible, will be mailed to you at the mailing address you designated within a week after completing the study. This is optional if you do not want to provide your email address or a mailing address. \par

\textbf{Who can I talk to?} \par

Please feel free to ask questions regarding this study. You may contact me later if you have additional questions or concerns at 1-979-845-5284 and mapalma@tamu.edu, Dr. Marco Palma. You may also contact the Human Research Protection Program at Texas A\&M University (which is a group of people who review the research to protect your rights) by phone at 1-979-458-4067, toll-free at 1-855-795-8636, or by email at irb@tamu.edu for:\par

\begin{itemize}
    \item Additional help with any questions about the research.
    \item Voicing concerns or complaints about the research.
    \item Obtaining answers to questions about your rights as a research participant.
    \item Concerns in the event the research staff could not be reached.
    \item The desire to talk to someone other than the research staff.
\end{itemize} \par



If you want a copy of this consent for your records, you can print it from the screen. \par

I freely agree to participate in this study \par
[Dropdown menu options:] \par
Yes, I agree \par

No, I don't agree and want to terminate the study

\section{Online experiment}

 Please indicate how often you consume oranges and similar fruits like tangerines, grapefruits, and mandarins (Screen out if last two categories).  

\begin{itemize}
    \item Daily
    \item 3-4 times a week
    \item Once a week
    \item 2-3 times a month
    \item Once a month
    \item Rarely or Never
\end{itemize}

\vspace{1cm} % Adds spacing before the warning

\textbf{Warning:} This study contains quality checks. You might be excluded at any point of the study if you do not pass quality checks or if you are not paying attention.


\clearpage


\subsection{BDM Hypothetical}

\subsubsection*{Instructions }


For this study, we will explore your preferences for oranges. \textbf{You will only receive a participation fee of \$2.00}. \par

 In the next screen you will be presented with a hypothetical scenario. Note that the scenario is hypothetical, but we ask you to make your decisions as if you were in a real market using real money. \par

Although we will describe the rules of the procedure shortly as if it actually involves money, you will not actually have to pay anything, and we will not ship any products to you. \par

However,\textbf{ we do want you to treat it as if it was real}. That is, although you will not receive a bag of oranges and you will not actually have to pay anything, we still want you to formulate your bid thinking what you would do if you had to do it for real. \par

You will receive more details about this procedure below.

\clearpage



\subsubsection*{Instructions (continued)}

In this study, you will complete 3 Tasks and a survey.

For each task, you will be endowed with a card with a value of \$6.00. You can use the funds in the card to purchase a bag of oranges (3.2 lbs – approximately 6-8 oranges) in two distinct markets. Any remaining funds in the card that you don’t spend on the oranges will be added to your earnings on top of your \$2.00 participation fee if you are selected for payment.



In one market, the oranges are from trees treated with antibiotics by injection to the *tree’s trunks* (not directly in the orange fruits) to prevent yield and fruit quality losses caused by a disease called citrus greening. The other market is conventional oranges, not from trees treated with antibiotics, but using a combination of pesticides and cultural practices.

You can use the funds in your card to bid on one of the markets or both. Your bid will be compared to an unrelated fixed price that is equally likely to be a number between $0.00 and $6.00. If your bid is higher than the fixed price, you will purchase the item. But here is the interesting part, in this case, you do not pay the price you offer, instead you pay the fixed offer. If your bid is lower than the fixed price, you do not buy the orange and you do not pay anything. 

\vspace{0.5cm}

{\textbf{\uline{Please remember this decision is hypothetical
and you will not earn money  or any bag of oranges.}}}



In each market (see below), you can choose the maximum amount you are willing to pay for a bag of oranges.
\clearpage

\subsubsection{\textbf{Instructions (continued)}}
 Here is a practice screen to familiarize yourself with the slider. Your responses here will not count toward the main study.

\begin{figure}[H]
    \centering
    \includegraphics[width=0.8\linewidth]{BDM_market.jpg}

\end{figure}

\clearpage

\definecolor{navyblue}{RGB}{0, 0, 128}

\subsubsection*{Instructions (continued)}
Let me give you an example: Assume that you bid \textbf{\$3.00}, and the fixed price is \textcolor{cyan}{\$2.80}, then you win the orange and only pay \textcolor{cyan}{\$2.80}. If, however, your price was lower than the fixed price, then you do not buy the orange.  

You should offer the maximum price you are willing to pay for the orange; there are no advantages to strategic behavior in this task. Your best strategy is to determine your personal value for the orange and offer that price.  

\vspace{0.3cm}

\textbf{Why is it my best strategy to bid the maximum price I’d be willing to pay?}  

If I truly value the orange at \textbf{\$3.00}, however instead of bidding \textcolor{orange}{\$3.00}, I bid \textcolor{orange}{\$2.60}, and the fixed price is \textcolor{cyan}{\$2.80}, then I lose the opportunity to purchase the oranges. However, had I bid my true valuation I would have won and paid only \textcolor{cyan}{\$2.80} for an orange I think is worth \textbf{\$3.00}.  

Now what if I offer more? Let’s say that I truly valued the orange at \textbf{\$3.00} and I bid \textcolor{orange}{\$5.00}, and the fixed price ended up being \textcolor{orange}{\$4.40}. Then I win, I will have to pay \textcolor{cyan}{\$4.40} for an item I value at \textbf{\$3.00}.  

\vspace{0.5cm} 

When you have understood the instructions, please proceed to the next page.  


\clearpage


\subsubsection*{Instructions continued}


\textbf{Example 1}: If the card you own can be redeemed for a bonus of \$6.00, you bid \$3.00 and the fixed price is \$2.40, you have a higher bid than the fixed price. You buy the item, and you pay the fixed price (\$2.40), plus your balance is (\$6.00 - \$2.40 = \$3.60) and you get the product which will be shipped to your specified mailing address using priority shipping.
\vspace{0.5cm}


\textbf{Example 2}: If the card you own can be redeemed for a bonus of \$6.00, your bid is \$3.00 and the fixed price is \$4.00, you have a bid lower than the fixed price. You do not buy the item, and you receive your bonus of \$6.00.

\vspace{0.5cm}

When you have understood the instructions, please proceed to the next page.
\clearpage


\subsubsection*{Comprehension test}
Note: All following questions will provide simple explanations if the subject indicates or types the wrong answer(correct answer is in bold).

\vspace{0.5cm}

True or False? The fixed price will be a known number to me before I make a decision.  

• True  \par
\textbf{• False}\par  

\vspace{0.5cm}


\vspace{0.5cm}



True or False? My best strategy is to bid the maximum I'd be willing to pay.  

\textbf{• True}  \par
• False 

\vspace{0.5cm}

If I value both oranges at \$5.00:

\textbf{a) Although the value of my card is less than \$10.00, I can bid \$5.00 in each market. However, only one of these bids will be binding.}  

b) I can only bid in one market since the value of my card is less than \$10.00.  

\vspace{0.5cm}

If the card you own could be redeemed for a bonus of \$6.00, your bid was \$4.00 and the fixed price was \$2.4, what would have been your final outcome?  

\textbf{a) I pay \$2.40 for the orange, so my balance is \$6 - \$2.40 = \$3.60 } 

b) The value of my card is \$6.00, since I cannot buy the orange  

c) I pay \$4.00 for the orange, so my balance is \$6.00 - \$4.00 = \$2.00  

\vspace{0.5cm}

If the card you own could be redeemed for a bonus of \$6.00, your bid is \$3.40 and the fixed price is \$4.00, what would be your final outcome?  

d) I pay \$4.00 for the orange, so my balance is \$6.00 - \$4.00 = \$2.00  

\textbf{e) The value of my card is \$6.00, since I cannot buy the orange } 

f) I pay \$3.4 for the orange, so my balance is \$6.00 - \$3.40 = \$2.60  

\vspace{0.5cm}

\subsubsection*{\textbf{Instructions (continued)}}

You have successfully completed the comprehension test, and you will now begin Task 1.\par

\vspace{0.5cm}
 Please click next when you are ready to proceed.

Now, we will go to the main tasks. 

\textbf{Please remember this decision is hypothetical and you will not earn money or a bag of oranges.}

\clearpage

\subsubsection*{\centering \textbf{Round 1 \& 2}}

\begin{figure}[H]
    \centering
    \includegraphics[width=0.8\linewidth]{BDM_market.jpg}
    \caption{}
    \label{fig:BDM_market}
\end{figure}

\clearpage

\subsubsection*{\textbf{Instructions (continued)}}
On the next page, you will see a video. \textbf{Please ensure your speakers or headphones are connected and the volume is set properly.} Next you will see a video that provides information about a disease Watch the video carefully in a quiet, distraction-free environment, as it is essential for the study.
\vspace{0.5cm}
Click on the video to watch it.

\href{https://www.youtube.com/watch?v=_AqMBjB0ChM}{video}
\clearpage


 \subsubsection*{\centering Round 3}


\begin{figure}[H]
    \centering
    \includegraphics[width=\linewidth]{BDM_market.jpg}
    \caption{}
    \label{fig:BDM_market}
\end{figure}

\clearpage



\subsection{BDM Real}


\subsubsection*{\textbf{Instructions (continued)}}


For this study you will receive a fixed fee of \$2 and you can earn additional compensation in cash and/or citrus products based on your decisions and chance, so please pay attention to the instructions.
 
You will complete 3 Tasks that involve real money, so please pay attention to the instructions. There is a 10\% chance that your decisions will be selected for payment. After completing the 3 tasks, the computer will draw a random number between 1 and 10. If the random number is equal to 1, you will receive the payoff as described below. However, if the random number is between 2 and 10, your decision will not yield additional earnings, and you will receive just the \$2. If you are selected for payment, one of the three tasks will be randomly selected to be added to your compensation.

 Since there is a chance that your decision will end up affecting your earnings, it is in your best interest to respond carefully to each task. If your compensation includes a food product (oranges), the product will be shipped to your specified address *free of charge* via priority mail as long as you provide us with a valid mailing address. You will receive more details about this procedure below.

\clearpage

\subsubsection*{\textbf{Instructions (continued)}}

In this study, you will complete 3 Tasks and a survey.
For each task, you will be endowed with a card with a value of \$6.00. You can use the funds in the card to purchase a bag of oranges (3.2 lbs – approximately 6-8 oranges) in two distinct markets. Any remaining funds in the card that you don’t spend on the oranges will be added to your earnings on top of your \$2.00 participation fee if you are selected for payment.



In one market, the oranges are from trees treated with antibiotics by injection to the *tree’s trunks* (not directly in the orange fruits) to prevent yield and fruit quality losses caused by a disease called citrus greening. The other market is conventional oranges, not from trees treated with antibiotics, but using a combination of pesticides and cultural practices.


You can use the funds in your card to bid on one of the markets or both. Your bid will be compared to an unrelated fixed price that is equally likely to be a number between $0.00 and $6.00. If your bid is higher than the fixed price, you will purchase the item. But here is the interesting part, in this case, you do not pay the price you offer, instead you pay the fixed offer. If your bid is lower than the fixed price, you do not buy the orange and you do not pay anything. 

 

\textbf{Please remember this decision is real and you may earn money or any bag of oranges.}

 

 \clearpage

\subsubsection*{\textbf{Instructions (continued)}}

From that point, the instructions are the same as hypothetical, with a different reminder as shown below;

 Now, we will go to the main tasks.


\textbf{Please remember this decision is real and you may earn money or any bag of oranges.}

 \clearpage

 \subsection{GSO Hypothetical}
 \subsubsection*{\textbf{Instructions (continued)}}

For this study, we will explore your preferences for oranges. You will only receive a participation fee of \$2.00. In the next screen you will be presented with a hypothetical scenario. Note that the scenario is hypothetical, but we ask you to make your decisions as if you were in a real market using real money. Although we will describe the rules of the procedure shortly as if it actually involves money, you will not actually have to pay anything, and we will not ship any products to you. However, we do want you to treat it as if it was real. That is, although you will not receive a bag of oranges and you will not actually have to pay anything, we still want you to formulate your bid thinking what you would do if you had to do it for real. You will receive more details about this procedure below.


\clearpage

\subsubsection*{\textbf{Instructions (continued)}}

In this study, you will complete 3 Tasks and a survey.

For each task, you will be endowed with a card with a value of \$6.00. You can use the funds in the card to purchase a bag of oranges (3.2 lbs – approximately 6-8 oranges) in two distinct markets. Any remaining funds in the card that you don’t spend on the oranges will be added to your earnings on top of your \$2.00 participation fee if you are selected for payment.

In one market, the oranges are treated with antibiotics by injection to the *tree’s trunks* (not directly in the orange fruits) to prevent yield and fruit quality losses caused by a disease called citrus greening. The other market is conventional oranges, not treated with antibiotics, but using a combination of pesticides and cultural practices.

You can use the funds in your card to purchase from one of the markets or both.

 
\textbf{Please remember this decision is hypothetical and you will not earn money or any bag of oranges.}

\clearpage

\subsubsection*{\textbf{Instructions (continued)}}

 In each market, offer prices will be displayed on the screen, and you will make choices over several periods. For every period, you can click “Try to Buy” if you are willing to buy the bag of oranges at the corresponding offer price or click “Not Buy” if you are not willing to buy at that offer price.
The computer has already randomly chosen the \textbf{market price} at which it will accept to sell the bag of oranges to you. That price is drawn between \$0.00 and \$6.00 and is equally likely.

In the first period, the offer price starts at \textcolor{orange}{\$0.00 }and will increase by \$0.20 after every period. The task ends when the price on the screen increases to the \textbf{market price} that the computer chose to sell the bag of oranges.
At every other price, whether you choose “Try to Buy” or “Not Buy”, you will go to the next period and again choose between “Try to Buy” or “Not Buy”.
The task will continue until the offer price reaches the market price the computer chose to try to sell the bag of oranges to you. At that price:

\begin{enumerate}
    \item If you choose “Try to Buy”, you receive the market price, and task ends.
    \item If you choose “Not Buy” you will not be able to buy the bag of oranges and task ends.
\end{enumerate}


For example, let’s say you are in Period 1, when the offer price is \$0.00. If the computer tries to sell you the bag of oranges at that price, and you click “Try to Buy”, you will get the bag of oranges at the current price \$0.00. If instead you clicked “Not Buy”, you will not be able to buy the bag of oranges. However, if the computer does not try to sell you the bag of oranges at that price, you will go to the next period.

\clearpage



\subsubsection*{\textbf{Instructions (continued)}}


Here is a practice screen to familiarize yourself with the buttons or with the way you choose to buy or not buy. Your responses here will not count toward the main study.
Please keep making choices until the buttons turn red, that is when then the market is over.

\begin{figure}[H]
    \centering
    \includegraphics[width=0.8\linewidth]{GSO.JPG}
    
    \label{fig:GSO}
\end{figure}

\clearpage


\subsubsection*{\textbf{Instructions (continued)}}

\textbf{Example 1}

Suppose the maximum amount \textbf{Person X} is willing to pay for the bag of oranges is \textbf{\$3.00}. The first offer price starts at \textcolor{orange}{\$0.00}. Since Person X is willing to purchase the bag of oranges for \textcolor{orange}{\$0.00} (because Person X is willing to pay up to \$3.00), then He/ She would click “Try to Buy” at that offer price.

Suppose the randomly drawn \textbf{market price} from \$0.00 to \$6 is \$2.00 (you will not be able to see it), since the offer price is lower than the market price, Person X would not be able to buy the bag of oranges when selecting “Try to Buy”. He/ She will proceed to the next period where offer price will be \textcolor{orange}{\$0.20}.
Since Person X is willing to purchase the bag of oranges at \$0.20, He/ She would select “Try to Buy”. Once again, the offer price will be compared to the market price. Since the randomly drawn market price in our example was \$2.00, Person X would not buy the bag oranges and He/ She will proceed to the next offer price of \textcolor{orange}{\$0.40}.


This process will continue until the offer price is \textcolor{orange}{\$2.00}. When the offer price is \textcolor{orange}{\$2.00}, Person X chooses “try to buy” and since the market price is \textcolor{cyan}{\$2.00} a transaction will occur at that time. Note that Person X maximum willingness to pay for the bag of oranges was \textbf{\$3.00}, but the market price randomly generated by the computer was \textcolor{cyan}{\$2.00}. So, if selected for payment, Person X would purchase the bag of oranges and receive them via priority shipping to their designated mailing address and the remaining \$4.00 on their card will be added to their compensation.



Please proceed to the next page for another example.
\clearpage

\subsubsection*{\textbf{Instructions (continued)}}

Example 2

Suppose the maximum amount \textbf{Person X} is willing to pay for the bag of oranges is \textbf{\$3.00}. The first offer price starts at \textcolor{orange}{\$0.00}. Since Person X is willing to purchase the bag of oranges for \textcolor{orange}{\$0.00} (because Person X is willing to pay up to \$3.00), then He/ She would click “Try to Buy” at that offer price.

Suppose the randomly drawn market price from \$0.00 to \$6.00 is \textcolor{cyan}{\$4.00} (you will not be able to see it), since the offer price is lower than the market price, Person X would not be able to buy the bag of oranges when selecting “Try to Buy”. He/ She will proceed to the next period where offer price will be \textcolor{orange}{\$0.20}.

Since Person X is willing to purchase the bag of oranges at \textcolor{orange}{\$0.20}, He/ She would select “Try to Buy”. Once again, the offer price will be compared to the market price. Since the randomly drawn market price in our example was \textcolor{cyan}{\$4.00}, Person X would not buy the bag oranges, and He/ She will proceed to the next offer price of \textcolor{orange}{\$0.40}.


This process will continue until the offer price is \textcolor{orange}{\$3.00}.  However, in the next round, the offer price will be \textcolor{orange}{\$3.20}. When the offer price is \textcolor{orange}{\$3.20}, Person X would click “Not buy”. When clicking “Not buy”, she will go to the next round and will continue to click “Not Buy” until task ends and will not be able to buy the bag of oranges. In this example, task ends when the offer price reaches \textcolor{cyan}{\$4.00}. Person X will click “Not Buy” at this price and will not be able to buy the bag of oranges.
\vspace{0.5cm}

When you have understood the instructions, please proceed to the next page.
\clearpage

\subsubsection*{Comprehension test}

Select which statement below is the most correct

\begin{itemize}
    \item The price increases by \$0.20 in each subsequent period 
    \item The value of your card does not change
    \item \textbf{Both answers are true}
\end{itemize}

\vspace{0.5cm}

What will happen to your reward if you try to buy when the fixed price is \$4.40, and the computer accepts your price?

\begin{itemize}
    \item I will get \$6, which is the value of my card
    \item \textbf{I will pay the oranges at \$4.40, plus get my balance of \$6.00-4.40=\$1.60}
    \item It is random
\end{itemize}

\vspace{0.5cm}


If the price of oranges in both markets is \$5.00 at a given period, can I try to buy both oranges in both markets?

\begin{itemize}
    \item No, you can’t because your card is worth \$6, it exceeds your endowment.
    \item \textbf{Yes, you can, however if the computer accepts both prices, only one decision will be randomly binding for realization.}
\end{itemize}

\vspace{0.5cm}

If Person' X maximum willingness to pay for the orange is \$3.00, S/He would switch from "Try to Buy" to "Not Buy" at \$3.20.

\begin{itemize}
    \item \textbf{Then Person X will continue to click "Not Buy" until the market ends}
    
    \item Person X should try to switch back and forth to "Try to Buy" and "Not Buy"
    \item None of these answers is correct
\end{itemize}

\clearpage



\subsubsection*{Comprehension test (continued)}

You have successfully completed the comprehension test, and you will now begin Task 1.

 Please click next when you are ready to proceed.

 Now, we will go to the main tasks. 
 
 \textbf{Please remember this decision is hypothetical and you will not earn money or any bag of oranges.}


\clearpage

 \subsubsection*{Round 1\& 2}


 \begin{figure}[H]
    \centering
    \includegraphics[width=0.8\linewidth]{GSO.JPG}
    
    \label{fig:GSO}
\end{figure}

 \vspace{0.5cm}


On the next page, you will see a video. \textbf{Please ensure your speakers or headphones are connected and the volume is set properly.} Next you will see a video that provides information about a disease Watch the video carefully in a quiet, distraction-free environment, as it is essential for the study.
\vspace{0.5cm}
Click on the video to watch it.

\href{https://www.youtube.com/watch?v=_AqMBjB0ChM}{video}
\clearpage


 \subsubsection*{Round 3}



\begin{figure}[H]
    \centering
    \includegraphics[width=\linewidth]{GSO.JPG}
    \caption{}
    \label{fig:Appendix_GSO_game}
\end{figure}
 

\clearpage



\subsection{GSO Real}

\subsubsection*{Instructions}

For this study you will receive a fixed fee of \$2.00 and you can earn additional compensation in cash and/or citrus products based on your decisions and chance, so please pay attention to the instructions.
 
You will complete 3 Tasks that involve real money, so please pay attention to the instructions. There is a 10\% chance that your decisions will be selected for payment. After completing the 3 tasks, the computer will draw a random number between 1 and 10. If the random number is equal to 1, you will receive the payoff as described below.
However, if the random number is between 2 and 10, your decision will not yield additional earnings, and you will receive just the \$2.00. If you are selected for payment, one of the three tasks will be randomly selected to be added to your compensation.

\textbf{Since there is a chance that your decision will end up affecting your earnings, it is in your best interest to respond carefully to each task}. If your compensation includes a food product (oranges), the product \textbf{will be shipped to your specified address *free of charge* via priority mail} as long as you provide us with a valid mailing address. You will receive more details about this procedure below.


\clearpage

\subsubsection*{Instructions (continued)}

In this study, you will complete 3 Tasks and a survey.

For each task, you will be endowed with a card with a value of \$6.00. You can use the funds in the card to purchase a bag of oranges (3.2 lbs – approximately 6-8 oranges) in two distinct markets. Any remaining funds in the card that you don’t spend on the oranges will be added to your earnings on top of your \$2.00 participation fee if you are selected for payment.

In one market, the oranges are from trees treated with antibiotics by injection to the *tree’s trunks* (not directly in the orange fruits) to prevent yield and fruit quality losses caused by a disease called citrus greening. The other market is conventional oranges, not from trees treated with antibiotics, but using a combination of pesticides and cultural practices.

You can use the funds in your card to purchase from one of the markets or both.

\textbf{ Please remember this decision is real and you may earn money or any bag of oranges.}

\clearpage

\subsubsection*{Instructions (continued)}

 From that point, the instructions are the same as hypothetical, with a different reminder as shown below;

 Now, we will go to the main tasks.
\vspace{0.5cm}

\textbf{Please remember this decision is real and you may earn money or any bag of oranges.}


 \clearpage

\section{Background on citrus greening}

 The United States, which was once the world leader in citrus production, producing 50\% of the world's citrus in 1970, has experienced a decrease in its global share to 25\% in 2000, then to 5\% in 2023 \citep{munch_us_2023} . As a result of this staggering decline in domestic production, the country relies more on imports from countries such as Mexico and Chile. The recent decline in citrus production over the past 15 years in the US is mainly a result of a bacterial disease known as "Huanglongbing" (HLB) or citrus greening. This disease was first discovered in Florida in 2005, it accounts; for the infection of 80\% of citrus plants in Florida \citep{li2020citrus} and increases  production cost of farmers \citep{roka2009citrus} . This disease, caused by the bacteria "Liberibacter asiaticus", results in chlorosis of the foliage, reduction of absorption of nutrients and yield, and death of citrus trees within 3 years \citep{bove_huanglongbing_2006}. The citrus greening also affects the marketability of the citrus by distorting its size, visual feature, ripening (partial or not at all), and flavor \citep{farnsworth_potential_2024}. Orange juice made from infected citrus fruit has a higher concentration of limonin and nomilin leading to a sour taste of citrus juice \citep{paula2018active}. 

Previously, to address HLB, systematic tree removal was recommended to combat HLB. However, this approach was not sustainable for multiple reasons. First, symptomatic citrus trees could still produce some marketable fruit, making immediate tree removal financially cumbersome for farmers. Second, replacing an infected tree with a new one incurs high costs for farmers, as it takes several years before new trees begin to bear fruit.  Finally, the effect of tree removal is undermined by the external cost. If some farmers decide not to remove infected trees, the disease contributes to spreading, negating the effort of others \citep{farnsworth_potential_2024}. Recently, the administration of antibiotics through trunk injections has been investigated and found to be a promising method to control HLB \citep{li_precision_2022}. \citet{archer_trunk_2023} corroborates this discovery by showing that the administration of oxytetracyclin through injection into the trunk of orange trees results in a decrease in fruit loss before harvest, as well as an improvement in fruit output, size, and juice quality.

Antibiotics administration has been applied for decades in cattle. However, according to \citet{hosain_antimicrobial_2021}, it has drawbacks on human health; it accounts for 80\% of antimicrobial resistance (AMR) in USA annually, and AMR affects 2.8 million people, causing more than 35.000 deaths \citep{cdc2019antibiotic}.  The use of antibiotics can effectively help to manage the dissemination of HLB, thus protecting citrus trees. Consequently, this can be advantageous to citrus farmers and all other stakeholders involved in the citrus industry at the state level in Florida. Antibiotics can help decrease Florida's reliance on imported citrus, thus preventing the probable extinction of citrus at the state level. This is undesirable for those who enjoy Florida citrus fruits and juice, as well as those who prefer locally sourced products. Ultimately, it can help preserve the superior taste of juice, preventing the occurrence of a bitter flavor that may arise from citrus fruits obtained from infected trees. The use of antibiotics also comes with some challenges. First, it may cause health hazards (AMR) if relevant policy is not reinforced on dosage and injection timing. Second, it may leave  residues in wastewater that could lead to soil pollution. In light of this dilemma, it directs our attention to determining how consumers will respond to the use of antibiotics in the production of citrus fruit since the success of such techniques depends on the acceptance of consumers, which also depends on the way information is communicated. Although the use of antibiotics in livestock has been extensively studied, it is unfortunate that the literature is silent about consumers’ preference for foods treated with antibiotics; hence, we propose filling this gap.

\clearpage










\section{Tables for main results}

\begin{table}[htbp]
\centering
\footnotesize
\caption{Standardized Differences comparing participants included vs excluded in the study}
\label{tab:Incomplete}
\begin{threeparttable}
\begin{tabular}{llccc}
\toprule
\textbf{Variable} & \textbf{Category} & \textbf{Include} & \textbf{Exclude} & \textbf{Std. Diff} \\
\midrule
\multirow{2}{*}{Ethnicity} & Non-White & 413 (34.9\%) & 345 (31.1\%) & 0.082 \\
                           & White     & 771 (65.1\%) & 766 (68.9\%) & \\
\midrule
\multirow{3}{*}{Income Category} & Low income    & 537 (46.3\%) & 535 (50.9\%) & 0.147 \\
                                 & Middle income & 519 (44.7\%) & 397 (37.8\%) & \\
                                 & High income   & 104 (9.0\%)  & 119 (11.3\%) & \\
\midrule
\multirow{4}{*}{Region} & Northeast & 200 (16.9\%) & 196 (17.7\%) & 0.074 \\
                        & Midwest   & 265 (22.4\%) & 259 (23.4\%) & \\
                        & South     & 464 (39.3\%) & 397 (35.8\%) & \\
                        & West      & 252 (21.3\%) & 257 (23.2\%) & \\
\midrule
\multirow{2}{*}{Gender} & Male   & 573 (48.4\%) & 523 (47.1\%) & 0.062 \\
                        & Female & 593 (50.1\%) & 562 (50.6\%) & \\
\midrule
\multirow{2}{*}{Education} & No Bachelor & 751 (63.4\%) & 701 (63.1\%) & 0.007 \\
                           & Bachelor    &             &             & \\
\midrule
\multirow{2}{*}{Children} & Children     & 405 (35.0\%) & 207 (19.8\%) & 0.346 \\
                          & No Children  & 765 (64.6\%) & 746 (67.1\%) & \\
\midrule
\multirow{2}{*}{Marital Status} & Married     & 419 (35.4\%) & 365 (32.9\%) & 0.053 \\
                                & Not Married &             &             & \\
\midrule
Age  & & 46 (16.53) & 53.32 (17.96) & -0.424 \\
\bottomrule
\end{tabular}
\begin{tablenotes}
\footnotesize
\item Notes: Standardized differences are calculated between those included and excluded in the study. In parentheses, proportions for categorical variables and standard deviations for continuous ones.
\end{tablenotes}
\end{threeparttable}
\end{table}

\clearpage


        % Right column for the second table

            \begin{table}[htbp]
                \centering
                \caption{Descriptive Statistics by Treatment 
                (excluding excluding MS (R1: 14.31\%, R2, 9.6\%, R3: 3.87\%)
)}
                \label{tab:Appendix_descrip_2}
                \resizebox{1\textwidth}{!}{%
                \begin{tabular}{lccc}
                \hline \hline
                \textbf{Treatment}    & \textbf{N (including MSB)}       & \textbf{Mean (USD)}  & \textbf{Standard Deviation} \\
                \hline
                GSO Real           & 288                        & 3.50       & 1.20                  \\
                BDM Real            & 313                        & 3.48       & 1.36                \\
                GSO Hypothetical     & 295                     & 3.63      & 1.16                   \\
                BDM Hypothetical      & 320                  & 3.44       & 1.30                    \\
                \midrule
                \textbf{Total}         & \textbf{1,216}              & \textbf{3.48} & \textbf{1.26}         \\
                \hline\hline
              
                \end{tabular}}
\begin{tablenotes}
            \footnotesize
            
            \item Notes: Mean and standard deviation are computed excluding \hl{???}. GSO stands for the Game Structure Obvious mechanism treatment; BDM stands for the Becker-DeGroot-Marschak mechanism treatment; Real and Hypo stand for the real and hypothetical incentives treatments, respectively.
           % \item $\times$: interaction
           % \item MS: A dummy variable for MS
           % \item GSOMSB: A dummy variable for being GSO and with MS
        \end{tablenotes}
            \end{table}
       
     
\clearpage
  









    \begin{table}[htbp!]
\centering
\caption{Standardized Differences Across Variables}
\label{tab:Appendix_std_diff_table}
\resizebox{\textwidth}{!}{%
\begin{tabular}{lcccccc}
\hl& \multicolumn{3}{c}{\textbf{GSO Real vs.}} & \multicolumn{2}{c}{\textbf{BDM Real vs.}} & \textbf{GSO Hypo vs.} \\
\cmidrule(lr){2-4} \cmidrule(lr){5-6} \cmidrule(lr){7-7}
\textbf{Variable}        & \textbf{BDM Real} & \textbf{GSO Hypo} & \textbf{BDM Hypo} & \textbf{GSO Hypo} & \textbf{BDM Hypo} & \textbf{BDM Hypo} \\
\hline

Hispanic        & 0.0884        & 0.0114        & 0.0630        & 0.0770        & 0.0254        & 0.0516 \\
Income\_category        & 0.1704        & 0.0462        & 0.1440        & 0.1498        & 0.0431        & 0.1153 \\
region  & 0.1326        & 0.1861        & 0.1918        & 0.1064        & 0.0779        & 0.0850 \\
gender  & 0.0834        & 0.0178        & 0.0228        & 0.0927        & 0.0627        & 0.0379 \\
Education       & 0.2267        & 0.1748        & 0.1342        & 0.0514        & 0.0919        & 0.0404 \\
children        & 0.0584        & 0.0128        & 0.0869        & 0.0456        & 0.0284        & 0.0740 \\
Marital_status  & 0.0952        & 0.0322        & 0.0437        & 0.1274        & 0.0514        & 0.0759 \\
Age     & -0.0160       & 0.0753        & -0.0557       & 0.0883        & -0.0380       & -0.1285 \\

\hline \hline
\end{tabular}}
\begin{tablenotes}
\item Notes: GSO stands for the Game Structure Obvious mechanism treatment; BDM stands for the Becker-DeGroot-Marschak mechanism treatment; Real and Hypo stand for the real and hypothetical incentives treatments, respectively.
\end{tablenotes}
\end{table}
  
 
\clearpage


 \begin{table}[H]
                \centering
                \caption{Interval Model of Consumer WTP for Oranges (Baseline: BDM Real)}
                \label{tab:interval_regression}
                \resizebox{\textwidth}{!}{% Scale table to fit the column width
    \begin{tabular}{l*{4}{cc}}
    \hline \hline
            &\multicolumn{2}{c}{WTP}    &\multicolumn{2}{c}{WTP(control 1)}    &\multicolumn{2}{c}{WTP(control 2}    &\multicolumn{2}{c}{WTP(control 3)}    \\
    \hline        
Constant    &       3.475\sym{***}&     (0.047)&       3.475\sym{***}&     (0.047)&       3.233\sym{***}&     (0.142)&       3.186\sym{***}&     (0.141)\\
GSO         &       0.128         &     (0.078)&                     &            &       0.095         &     (0.077)&                     &            \\
Hypothetical&      -0.026         &     (0.062)&      -0.026         &     (0.062)&      -0.037         &     (0.061)&      -0.037         &     (0.061)\\
GSO$\times$Hypothetical&       0.224\sym{**} &     (0.103)&                     &            &       0.298\sym{***}&     (0.103)&                     &            \\
GSOnoMSB  &                     &            &       0.055         &     (0.079)&                     &            &       0.018         &     (0.080)\\
GSOnoMSB$\times$Hypothetical&                     &            &       0.191\sym{*}  &     (0.106)&                     &            &       0.269\sym{**} &     (0.108)\\
GSOMSB    &                     &            &       1.012\sym{***}&     (0.148)&                     &            &       1.055\sym{***}&     (0.156)\\
GSOMSB$\times$Hypothetical&                     &            &       0.334\sym{*}  &     (0.187)&                     &            &       0.359\sym{*}  &     (0.189)\\
Round2      &                     &            &                     &            &       0.050\sym{*}  &     (0.026)&       0.060\sym{**} &     (0.026)\\
Round3      &                     &            &                     &            &       0.084\sym{***}&     (0.030)&       0.111\sym{***}&     (0.030)\\
Risk preferences&                     &            &                     &            &      -0.119\sym{***}&     (0.027)&      -0.120\sym{***}&     (0.027)\\
White       &                     &            &                     &            &      -0.056         &     (0.061)&      -0.057         &     (0.060)\\
Middle Income&                     &            &                     &            &      -0.047         &     (0.063)&      -0.046         &     (0.062)\\
High Income &                     &            &                     &            &       0.052         &     (0.117)&       0.066         &     (0.116)\\
Midwest     &                     &            &                     &            &      -0.180\sym{**} &     (0.084)&      -0.186\sym{**} &     (0.082)\\
South       &                     &            &                     &            &      -0.092         &     (0.076)&      -0.093         &     (0.075)\\
West        &                     &            &                     &            &      -0.144\sym{*}  &     (0.085)&      -0.157\sym{*}  &     (0.084)\\
Female      &                     &            &                     &            &       0.044         &     (0.055)&       0.061         &     (0.055)\\
Bachelor    &                     &            &                     &            &      -0.082         &     (0.059)&      -0.084         &     (0.059)\\
No Children &                     &            &                     &            &      -0.012         &     (0.061)&       0.004         &     (0.060)\\
Married     &                     &            &                     &            &      -0.086         &     (0.062)&      -0.086         &     (0.061)\\
Age         &                     &            &                     &            &       0.009\sym{***}&     (0.002)&       0.010\sym{***}&     (0.002)\\

$\sigma_u $    &       1.090\sym{***}&     (0.020)&       1.079\sym{***}&     (0.019)&       1.077\sym{***}&     (0.019)&       1.064\sym{***}&     (0.019)\\
$\sigma_e $     &       0.847\sym{***}&     (0.019)&       0.838\sym{***}&     (0.019)&       0.844\sym{***}&     (0.019)&       0.835\sym{***}&     (0.019)\\
\hline
\(n (observations)\)       &        7296         &            &        7296         &            &        7110         &            &        7110         &            \\
\(N (subjects\)       &        1216         &            &        1216         &            &        1185         &            &        1185         &            \\
\hline \hline
\end{tabular}
}




\begin{tablenotes}
            \footnotesize
            \item Standard errors in parentheses. * p$<$0.1, ** p$<$0.05, *** p$<$0.01.
            \item \textit{Note:} $\sigma_u$ and $\sigma_e$ denote the standard deviations of the random intercept and residual error, respectively.
            \item Notes: GSO stands for the Game Structure Obvious mechanism treatment; BDM stands for the Becker-DeGroot-Marschak mechanism treatment; Real and Hypo stand for the real and hypothetical incentives treatments, respectively.
           \item $\times$: interaction
           \item GSOnoMSB: A dummy variable for being GSO and no MS.
           \item GSOMSB: A dummy variable for being GSO and with MS.
           \item Baseline: BDM Real. represents the baseline treatment group.
        \end{tablenotes}
            \end{table}








\clearpage


    \begin{table}[htbp!]
        \centering
        \caption{Two-Sample Test of Proportions (complexity above median)}
        \label{tab:Appendix_proportion_test}
        \resizebox{0.9\textwidth}{!}{% Scale table to fit the column width
        \begin{tabular}{lcccccc}
\hline\hline
\textbf{Group} & \textbf{Obs} & \textbf{Mean} & \textbf{Std. Err.} & \textbf{z} & \textbf{$P>|z|$} & \textbf{95\% Conf. Interval (Lower)}  \\
\hline
BDM            & 3,798        & 0.4186414     & 0.008           &            &                 & [0.40, 0.43]                             \\
GSO            & 2,376        & 0.36     & 0.008          &            &                 & [0.35, 0.38]                               \\
\hline
\textbf{Diff}  &              & 0.05    & 0.011          & 4.35     & 0.000           & [0.027, 0.072]                             \\
\hline\hline
        \end{tabular}
        }
    \end{table}


\clearpage

\begin{table}[htbp!]
    \centering
    \caption{Two-Sample Test of Proportions by Complexity above median (Group MSB vs. Non-MSB)}
    \label{tab:Appendix_proportion_test_groupMSB_vs_Non-MSB}
    \resizebox{0.9\textwidth}{!}{%
    \begin{tabular}{lcccccc}
\hline\hline
\textbf{Group} & \textbf{Obs} & \textbf{Mean} & \textbf{Std. Err.} & \textbf{z} & \textbf{$P>|z|$} & \textbf{95\% Conf. Interval (Lower)} \\
\hline
Non-MSB             & 6,174        & 0.3732        & 0.0062              &            &                  & [0.3611, 0.3852]                    \\
MSB              & 1,122        & 0.5134        & 0.0149              &            &                  & [0.4841, 0.5426]                    \\
\hline
\textbf{Diff}  &              & -0.1402       & 0.0161              & -8.84      & 0.000            & [-0.1718, -0.1086]                  \\
\hline\hline
    \end{tabular}
    }
\end{table}

\clearpage





 \begin{table}[H]
        \centering
        \caption{Comparison of Willingness to Pay by Instruction Comprehension Scores (aggregate) }
        \label{tab:MSB_interval_regression_Score_overall}
        \resizebox{0.7\textwidth}{!}{% Scale table to fit the column width
  \begin{tabular}{l*{2}{cc}}
  \hline \hline
            &\multicolumn{2}{c}{WTP (GSO)}    &\multicolumn{2}{c}{WTP (BDM)}    \\
            \hline
Constant    &       3.136\sym{***}&     (0.309)&       2.937\sym{***}&     (0.236)\\
Hypothetical&       0.026         &     (0.259)&       0.084         &     (0.174)\\
test\_score  &       0.039         &     (0.033)&       0.027         &     (0.021)\\
Hypothetical$\times$test\_score&       0.036         &     (0.041)&      -0.021         &     (0.027)\\
Round2      &       0.048         &     (0.044)&       0.064\sym{*}  &     (0.033)\\
Round3      &       0.096\sym{*}  &     (0.053)&       0.116\sym{***}&     (0.038)\\
Risk preferences&      -0.069         &     (0.042)&      -0.165\sym{***}&     (0.037)\\
GSOMSB    &       1.079\sym{***}&     (0.102)&                     &            \\
White       &      -0.077         &     (0.094)&      -0.037         &     (0.075)\\
Middle Income&      -0.047         &     (0.097)&      -0.063         &     (0.077)\\
High Income &       0.413\sym{**} &     (0.180)&      -0.153         &     (0.157)\\
Midwest     &      -0.152         &     (0.129)&      -0.258\sym{**} &     (0.112)\\
South       &      -0.091         &     (0.119)&      -0.125         &     (0.102)\\
West        &      -0.184         &     (0.136)&      -0.147         &     (0.116)\\
Female      &      -0.085         &     (0.084)&       0.154\sym{**} &     (0.067)\\
Bachelor    &      -0.240\sym{***}&     (0.093)&       0.073         &     (0.078)\\
No Children &       0.150         &     (0.101)&      -0.111         &     (0.082)\\
Married     &      -0.227\sym{**} &     (0.097)&       0.026         &     (0.079)\\
Age         &       0.009\sym{***}&     (0.003)&       0.010\sym{***}&     (0.002)\\
$\sigma_u$     &       1.118\sym{***}&     (0.031)&       1.006\sym{***}&     (0.028)\\
$\sigma_e $    &       0.812\sym{***}&     (0.031)&       0.841\sym{***}&     (0.023)\\
\hline
\(n\) (observations)       &        3408         &            &        3702         &            \\
\(N\) (subjects)       &        568         &            &        617         &            \\
\hline \hline
\end{tabular}
}


\begin{tablenotes}
            \footnotesize
            \item Standard errors in parentheses. * p$<$0.1, ** p$<$0.05, *** p$<$0.01.
            \item \textit{Note:} $\sigma_u$ and $\sigma_e$ denote the standard deviations of the random intercept and residual error, respectively.
            \item Notes: GSO stands for the Game Structure Obvious mechanism treatment; BDM stands for the Becker-DeGroot-Marschak mechanism treatment; Real and Hypo stand for the real and hypothetical incentives treatments, respectively.
           \item $\times$: Interaction.
           \item GSOMSB: A dummy variable for being GSO and with MS.
           \item Baseline: BDM Real. represents the baseline treatment group for the right hand-side. 
           GSO Real represents the Baseline treatment group for the left-hand side.
        \end{tablenotes}
\end{table}
\clearpage




 \begin{table}[H]
        \centering
        \caption{Comparison of Willingness to Pay by Instruction Comprehension Scores (Top and Bottom Quartiles) for GSO and BDM}
        \label{tab:MSB_interval_regression_BDM_Scorehigh_low}
        \resizebox{0.7\textwidth}{!}{% Scale table to fit the column width
     \begin{tabular}{l*{2}{cc}}
     \hline \hline

            &\multicolumn{2}{c}{WTP (GSO)}    &\multicolumn{2}{c}{WTP (BDM)}    \\
            \hline
Constant    &       3.227\sym{***}&     (0.266)&       2.995\sym{***}&     (0.234)\\
Hypothetical&       0.177         &     (0.126)&      -0.060         &     (0.094)\\
Top 25\%     &      -0.231         &     (0.221)&       0.087         &     (0.134)\\
Top25\%$\times$Hypothetical&       0.954\sym{***}&     (0.279)&       0.137         &     (0.178)\\
Round2      &       0.029         &     (0.059)&       0.092\sym{**} &     (0.038)\\
Round3      &       0.072         &     (0.062)&       0.132\sym{***}&     (0.046)\\
Risk preferences&      -0.033         &     (0.054)&      -0.189\sym{***}&     (0.045)\\
GSOMSB    &       1.179\sym{***}&     (0.147)&                     &            \\
White       &       0.053         &     (0.124)&      -0.043         &     (0.091)\\
Middle Income&      -0.073         &     (0.125)&      -0.206\sym{**} &     (0.091)\\
High Income &       0.483\sym{***}&     (0.180)&      -0.200         &     (0.185)\\
Midwest     &       0.031         &     (0.169)&      -0.111         &     (0.131)\\
South       &       0.075         &     (0.156)&      -0.006         &     (0.129)\\
West        &      -0.095         &     (0.183)&       0.044         &     (0.144)\\
Female      &      -0.078         &     (0.110)&       0.064         &     (0.079)\\
Bachelor    &      -0.265\sym{**} &     (0.126)&       0.157\sym{*}  &     (0.092)\\
No Children &       0.193         &     (0.126)&      -0.166\sym{*}  &     (0.095)\\
Married     &      -0.381\sym{***}&     (0.124)&       0.096         &     (0.094)\\
Age         &       0.008\sym{**} &     (0.004)&       0.011\sym{***}&     (0.003)\\
$\sigma_u$     &       1.083\sym{***}&     (0.041)&       0.972\sym{***}&     (0.033)\\
$\sigma_e$     &       0.815\sym{***}&     (0.041)&       0.822\sym{***}&     (0.023)\\
\hline
\(n\)       &        2118         &            &        2544         &            \\
\(N\)       &        353        &            &        424         &            \\
\hline \hline
\end{tabular}
}


\begin{tablenotes}
            \footnotesize
            \item Standard errors in parentheses. * p$<$0.1, ** p$<$0.05, *** p$<$0.01.
            \item \textit{Note:} $\sigma_u$ and $\sigma_e$ denote the standard deviations of the random intercept and residual error, respectively.
            \item Notes: GSO stands for the Game Structure Obvious mechanism treatment; BDM stands for the Becker-DeGroot-Marschak mechanism treatment; Real and Hypo stand for the real and hypothetical incentives treatments, respectively.
           \item $\times$: Interaction.
           \item GSOMSB: A dummy variable for being GSO and with MS.
           \item Baseline: BDM Real. represents the baseline treatment group for the right hand-side.
           GSO Real represents the Baseline treatment group for the left-hand side.
        \end{tablenotes}
\end{table}


\clearpage




\begin{table}[htbp!]
    \centering
    \caption{Game form recognition}
    \label{tab:Appendix_gameform}
    \resizebox{0.9\textwidth}{!}{%
    \begin{tabular}{lcccc}
    \toprule
    \textbf{Variable} & \textbf{Group} & \textbf{Mean} & \textbf{Std. Dev.} & \textbf{t-stat (p-value)} \\
    \midrule
    \text{Complexity score}
      & BDM & 2.33 & 1.06 & $(p < 0.01)$ \\
      & GSO & 2.19 & 1.06 & \\
    \midrule
    \text{Choice certainty}
      & BDM & 3.66 & 1.01 & \ $(p < 0.01)$ \\
      & GSO & 3.74 & 0.99 & \\
    \hline \hline
    \end{tabular}}
\end{table}

\clearpage



 \begin{table}[H]
        \centering
        \caption{Comparison of Willingness to Pay by Support for Antibiotic Use in Orange Production}
        \label{tab:interval_regression_socialdesirability_BDM}
        \resizebox{0.7\textwidth}{!}{% Scale table to fit the column width
        \begin{tabular}{l*{2}{cc}}
        \hline \hline
            &\multicolumn{2}{c}{WTP: GSO (Yes)}    &\multicolumn{2}{c}{WTP: BDM (Yes)}    \\
            \hline
Constant    &       3.398\sym{***}&     (0.241)&       3.396\sym{***}&     (0.111)\\
Hypothetical&       0.379\sym{***}&     (0.104)&       0.174\sym{***}&     (0.058)\\
Information&       0.160         &     (0.118)&       0.180\sym{**} &     (0.076)\\
Hypothetical$\times$Information&      -0.199         &     (0.172)&      -0.158         &     (0.110)\\
Risk preferences&      -0.003         &     (0.053)&       0.070\sym{**} &     (0.032)\\
GSOMSB      &       1.135\sym{***}&     (0.161)&                     &            \\
White&      -0.206\sym{*}  &     (0.111)&      -0.235\sym{***}&     (0.057)\\
Middle Income&      -0.425\sym{***}&     (0.108)&       0.155\sym{***}&     (0.059)\\
High Income&       0.175         &     (0.149)&       0.315\sym{***}&     (0.098)\\
Midwest   &      -0.018         &     (0.140)&      -0.149\sym{*}  &     (0.081)\\
South    &       0.186         &     (0.131)&      -0.011         &     (0.075)\\
West    &       0.123         &     (0.155)&      -0.309\sym{***}&     (0.081)\\
Female    &       0.067         &     (0.094)&       0.220\sym{***}&     (0.048)\\
Bachelor&       0.052         &     (0.115)&      -0.051         &     (0.057)\\
No Children  &       0.153         &     (0.101)&       0.156\sym{**} &     (0.067)\\
Maried&       0.012         &     (0.112)&      -0.233\sym{***}&     (0.052)\\
Age         &       0.004         &     (0.003)&       0.007\sym{***}&     (0.002)\\
$\sigma_u$     &       0.931\sym{***}&     (0.049)&       0.998\sym{***}&     (0.023)\\
$\sigma_e$     &       0.865\sym{***}&     (0.053)&       0.830\sym{***}&     (0.025)\\
\hline
\(n\) (observations)       &        1014         &            &        1224         &            \\
\(N\) (subjects)      &        169         &            &        204         &            \\
\hline \hline
\multicolumn{5}{l}{\footnotesize Standard errors in parentheses. * p$<$0.1, ** p$<$0.05 *** p$<$0.01}\\
\end{tabular}
}



\begin{tablenotes}
            \footnotesize
            \item Standard errors in parentheses. * p$<$0.1, ** p$<$0.05, *** p$<$0.01.
            \item \textit{Note:} $\sigma_u$ and $\sigma_e$ denote the standard deviations of the random intercept and residual error, respectively.
            \item Notes: GSO stands for the Game Structure Obvious mechanism treatment; BDM stands for the Becker-DeGroot-Marschak mechanism treatment; Real and Hypo stand for the real and hypothetical incentives treatments, respectively.
           \item $\times$: Interaction.
           \item GSOMSB: A dummy variable for being GSO and with MS
           \item Baseline: BDM Real. represents the baseline treatment group for the right hand-side \\
           GSO Real represents the Baseline treatment group for the left-hand side.
        \end{tablenotes}
\end{table}



\clearpage


 \begin{table}[H]
        \centering
        \caption{Comparison of Willingness to Pay by Type of Oranges}
        \label{tab:Orange_socialdesirability_BDM}
        \resizebox{0.7\textwidth}{!}{% Scale table to fit the column width
        \begin{tabular}{l*{2}{cc}}
        \hline \hline
            &\multicolumn{2}{c}{WTP: GSO }    &\multicolumn{2}{c}{WTP: BDM }    \\
            \hline
Constant    &       3.366\sym{***}&     (0.209)&       3.065\sym{***}&     (0.181)\\
Hypothetical&       0.246\sym{***}&     (0.093)&      -0.060         &     (0.069)\\
AntibioticsOranges&      -0.005         &     (0.063)&       0.083         &     (0.058)\\
Hypothetical$\times$Oranges&      -0.010         &     (0.086)&       0.045         &     (0.081)\\
Round2      &       0.048         &     (0.044)&       0.064\sym{*}  &     (0.033)\\
Round3      &       0.096\sym{*}  &     (0.053)&       0.116\sym{***}&     (0.038)\\
GSOMSB      &       1.078\sym{***}&     (0.101)&                     &            \\
Risk preferences&      -0.070\sym{*}  &     (0.042)&      -0.165\sym{***}&     (0.037)\\
White       &      -0.081         &     (0.094)&      -0.044         &     (0.075)\\
Middle Income&      -0.050         &     (0.096)&      -0.062         &     (0.077)\\
High Income &       0.385\sym{**} &     (0.178)&      -0.150         &     (0.157)\\
Midwest     &      -0.120         &     (0.129)&      -0.263\sym{**} &     (0.111)\\
South       &      -0.062         &     (0.118)&      -0.130         &     (0.101)\\
West        &      -0.178         &     (0.135)&      -0.155         &     (0.115)\\
Female      &      -0.080         &     (0.083)&       0.156\sym{**} &     (0.067)\\
Bachelor    &      -0.254\sym{***}&     (0.093)&       0.067         &     (0.078)\\
No Children &       0.148         &     (0.101)&      -0.107         &     (0.081)\\
Married     &      -0.234\sym{**} &     (0.096)&       0.025         &     (0.078)\\
Age         &       0.009\sym{***}&     (0.003)&       0.010\sym{***}&     (0.002)\\
$\sigma_u$     &       1.126\sym{***}&     (0.031)&       1.007\sym{***}&     (0.028)\\
$\sigma_e$     &       0.812\sym{***}&     (0.031)&       0.839\sym{***}&     (0.024)\\
\hline
\(n\) (observations)      &        3408         &            &        3702         &            \\
\(N\) (subjects)      &        169         &            &        204         &            \\
\hline \hline
\multicolumn{5}{l}{\footnotesize Standard errors in parentheses. * p$<$0.1, ** p$<$0.05 *** p$<$0.01}\\
\end{tabular}
}



\begin{tablenotes}
            \footnotesize
            \item Standard errors in parentheses. * p$<$0.1, ** p$<$0.05, *** p$<$0.01.
            \item \textit{Note:} $\sigma_u$ and $\sigma_e$ denote the standard deviations of the random intercept and residual error, respectively.
            \item Notes: GSO stands for the Game Structure Obvious mechanism treatment; BDM stands for the Becker-DeGroot-Marschak mechanism treatment; Real and Hypo stand for the real and hypothetical incentives treatments, respectively.
           \item $\times$: Interaction.
           \item GSOMSB: A dummy variable for being GSO and with MS
           \item Baseline: BDM Real. represents the baseline treatment group for the right hand-side \\
           GSO Real represents the Baseline treatment group for the left-hand side.
        \end{tablenotes}
\end{table}




\clearpage





\begin{table}[H]
        \centering
        \caption{Comparison of Willingness to Pay by MS in GSO}        \label{tab:MSB_interval_regression_MSB}
       \resizebox{0.7\textwidth}{!}{% Scale table to fit the column width
      \begin{tabular}{l*{1}{cc}}
      \hline \hline
            &\multicolumn{2}{c}{WTP}    \\
Constant    &       3.367\sym{***}&     (0.124)\\
\hline
Hypothetical&       0.232\sym{***}&     (0.056)\\
MSB      &       1.018\sym{***}&     (0.166)\\
Hypothetical$\times$MSB&       0.107         &     (0.211)\\
Round2      &       0.047         &     (0.049)\\
Round3      &       0.096\sym{*}  &     (0.054)\\
Risk preferences&      -0.070\sym{***}&     (0.023)\\
White&      -0.080         &     (0.052)\\
Midlle Income &      -0.051         &     (0.057)\\
High Income&       0.388\sym{***}&     (0.101)\\
Midwest    &      -0.120         &     (0.076)\\
South    &      -0.061         &     (0.064)\\
West    &      -0.177\sym{**} &     (0.077)\\
Female    &      -0.080\sym{*}  &     (0.048)\\
Bachelor&      -0.254\sym{***}&     (0.052)\\
No children  &       0.147\sym{**} &     (0.058)\\
Married&      -0.233\sym{***}&     (0.060)\\
Age         &       0.009\sym{***}&     (0.002)\\
$\sigma_u$     &       1.126\sym{***}&     (0.025)\\
$\sigma_e$     &       0.812\sym{***}&     (0.030)\\
\hline
\(n\) (observations)       &        3408         &            \\
\(N\) (subjects)       &        568         &            \\
\hline \hline
\end{tabular}
}

\begin{tablenotes}
            \footnotesize
            \item Standard errors in parentheses. * p$<$0.1, ** p$<$0.05, *** p$<$0.01.
            \item \textit{Note:} $\sigma_u$ and $\sigma_e$ denote the standard deviations of the random intercept and residual error, respectively.
            \item Notes: GSO stands for the Game Structure Obvious mechanism treatment; Real and Hypo stand for the real and hypothetical incentives treatments, respectively.
           \item $\times$: interaction
           \item GSOMSB: A dummy variable for being GSO and with MS
           \item Baseline: GSO Real represents the Baseline treatment group for the left-hand side
        \end{tablenotes}
\end{table}

\clearpage













\begin{table}[H]
    \centering
    \caption{Time spent across elicitation tasks}
    \label{tab:Appendix time}
\resizebox{\textwidth}{!}{%
\begin{tabular}{lcccc}
\hline \hline
\textbf{Statistic} & \multicolumn{2}{c}{\textbf{GSO }} & \multicolumn{2}{c}{\textbf{BDM}} \\
\cmidrule(lr){2-3} \cmidrule(lr){4-5}
& Hypothetical & Real & Hypothetical & Real \\
\midrule
n (observations)         & 1,403 & 1,323 & 1,920 & 1,878 \\
Mean                       &  3518.89 & 5483.53 & 4{,}107.2 & 4{,}8165 \\
Standard Error             & 154.32 & 969.10 & 296.37 & 484.66 \\
Standard Deviation         & 8.6  & 8.45 & 12{,}986.42 & 21{,}003.45 \\

Difference in Means        & \multicolumn{2}{c}{-1964.64} & \multicolumn{2}{c}{-709.05} \\
Std. Error of Difference   & \multicolumn{2}{c}{ 954.38} & \multicolumn{2}{c}{569.3} \\
$p$-value (two-sided)      & \multicolumn{2}{c}{ 0.0396} & \multicolumn{2}{c}{0.2048} \\
\hline\hline
\end{tabular}%
}
\end{table}

\clearpage

\begin{table}[H]
        \centering
        \caption{Comparison of Willingness to Pay by Round for GSO}
        \label{tab:interval_regression_RounbyRound}
        \resizebox{0.7\textwidth}{!}{% Scale table to fit the column width
     \begin{tabular}{l*{2}{cc}}
     \hline \hline
            &\multicolumn{2}{c}{WTP (GSO)}    &\multicolumn{2}{c}{WTP (BDM)}    \\
            \hline
Constant    &       3.329\sym{***}&     (0.208)&       3.114\sym{***}&     (0.182)\\
Hypothetical&       0.308\sym{***}&     (0.102)&      -0.053         &     (0.072)\\
Round2      &       0.103\sym{*}  &     (0.060)&       0.079\sym{*}  &     (0.046)\\
Round3      &       0.142\sym{**} &     (0.069)&       0.079         &     (0.054)\\
Hypothetical$\times$Round2&      -0.109         &     (0.092)&      -0.029         &     (0.061)\\
Hypothetical$\times$Round3&      -0.089         &     (0.101)&       0.074         &     (0.074)\\
Risk preferences&      -0.069\sym{*}  &     (0.042)&      -0.165\sym{***}&     (0.037)\\
GSOMSB    &       1.080\sym{***}&     (0.102)&                     &            \\
White       &      -0.080         &     (0.094)&      -0.044         &     (0.075)\\
Middle Income&      -0.050         &     (0.096)&      -0.062         &     (0.077)\\
High Income &       0.386\sym{**} &     (0.178)&      -0.150         &     (0.157)\\
Midwest     &      -0.119         &     (0.130)&      -0.263\sym{**} &     (0.111)\\
South       &      -0.062         &     (0.118)&      -0.130         &     (0.101)\\
West        &      -0.178         &     (0.135)&      -0.155         &     (0.115)\\
Female      &      -0.079         &     (0.083)&       0.156\sym{**} &     (0.067)\\
Bachelor    &      -0.253\sym{***}&     (0.093)&       0.067         &     (0.078)\\
No Children &       0.149         &     (0.101)&      -0.107         &     (0.081)\\
Married     &      -0.234\sym{**} &     (0.096)&       0.025         &     (0.078)\\
Age         &       0.009\sym{***}&     (0.003)&       0.010\sym{***}&     (0.002)\\
$\sigma_u$     &       1.126\sym{***}&     (0.031)&       1.007\sym{***}&     (0.028)\\
$\sigma_e$     &       0.812\sym{***}&     (0.031)&       0.840\sym{***}&     (0.023)\\
\hline
\(n\) (observations)      &        3408         &            &        3702         &            \\
\(N\) (subjects)      &        568         &            &        617         &            \\
\hline\hline
\end{tabular}
}



\begin{tablenotes}
            \footnotesize
            \item Standard errors in parentheses. * p$<$0.1, ** p$<$0.05, *** p$<$0.01.
            \item \textit{Note:} $\sigma_u$ and $\sigma_e$ denote the standard deviations of the random intercept and residual error, respectively.
            \item Notes: GSO stands for the Game Structure Obvious mechanism treatment; BDM stands for the Becker-DeGroot-Marschak mechanism treatment; Real and Hypo stand for the real and hypothetical incentives treatments, respectively.
           \item $\times$: Interaction.
           \item GSOMSB: A dummy variable for being GSO and with MS.
           \item Baseline: BDM Real. represents the baseline treatment group for the right hand-side.
           GSO Real represents the Baseline treatment group for the left-hand side.
        \end{tablenotes}
\end{table}





\clearpage



\begin{table}[H]
        \centering
        \caption{Time spent and hypothetical bias in GSO}        \label{tab:Time_spent}
       \resizebox{0.7\textwidth}{!}{% Scale table to fit the column width
      \begin{tabular}{l*{1}{cc}}
\hline\hline
            &\multicolumn{2}{c}{(1)}           \\
            &\multicolumn{2}{c}{Time spent (in milliseconds)}\\
\hline
Constant    &    7470.699\sym{***}&  (2570.860)\\
Hypothetical&   -4967.854\sym{*}  &  (3005.793)\\
Round2      &   -5289.760\sym{*}  &  (2956.289)\\
Round3      &   -5517.654\sym{*}  &  (2958.739)\\
Hypothetical$\times$Round2&    4235.668         &  (2971.166)\\
Hypothetical$\times$Round3&    4292.919         &  (2986.377)\\
Risk preferences&     612.794         &   (415.609)\\
White       &   -1541.616         &   (997.200)\\
Middle Income&    -670.852         &   (797.300)\\
High Income &   -1361.715         &   (983.571)\\
Midwest     &     860.712         &   (644.447)\\
South       &    1650.992         &  (1598.977)\\
West        &     696.555         &   (716.781)\\
Female      &   -1392.009         &   (969.674)\\
Bachelor    &    2904.234\sym{**} &  (1443.308)\\
No Children &   -1737.553         &  (1453.223)\\
Married     &    2019.505         &  (1357.798)\\
Age         &      35.408\sym{***}&    (12.654)\\
\hline
\(n (observations)\)       &        2662         &            \\
\hline\hline
\multicolumn{3}{l}{\footnotesize Standard errors in parentheses. * p$<$0.1, ** p$<$0.05 *** p$<$0.01}\\
\end{tabular}
}


\begin{tablenotes}
            \footnotesize
            \item Standard errors in parentheses. * p$<$0.1, ** p$<$0.05, *** p$<$0.01.
            \item Notes: GSO stands for the Game Structure Obvious mechanism treatment; Real and Hypo stand for the real and hypothetical incentives treatments, respectively.
           \item $\times$: interaction
           \item Baseline: GSO Real represents the Baseline treatment group for the left-hand side
        \end{tablenotes}
\end{table}

\clearpage





\clearpage
%%%%%%%%%%%%%%%%%%%%%%%%%%%%%%%%%%%%%%%%%%%%%%%%%%%%%%%%%%%%%%%%%%%%%%%%%%%%%%%%%%%%%%%%%%%%%%%%%%%%%%%%%%%%%%%%%%%%%%%%%%%%%%a%%%%%%
\section{Appendix sections including MS}




 \begin{table}[H]
                \centering
                \caption{Interval Model of Consumer WTP for Oranges (Baseline: BDM Real)}
                \label{tab:MSB_interval_regression_multiple_comaparisons}
                \resizebox{0.85\textwidth}{!}{% Scale table to fit the column width
             \begin{tabular}{l*{3}{cc}}
             \hline \hline
            &\multicolumn{2}{c}{WTP     (full sample)}    &\multicolumn{2}{c}{WTP (exclude MSB)}    &\multicolumn{2}{c}{WTP (only MSB)}    \\

Constant    &       3.187\sym{***}&     (0.141)&       3.197\sym{***}&     (0.144)&       3.106\sym{***}&     (0.180)\\
GSO         &       0.014         &     (0.077)&       0.015         &     (0.079)&                     &            \\
Hypothetical&      -0.037         &     (0.061)&      -0.037         &     (0.061)&      -0.038         &     (0.063)\\
GSOXHypothetical&       0.277\sym{***}&     (0.102)&       0.271\sym{**} &     (0.106)&                     &            \\
Round2      &       0.060\sym{**} &     (0.026)&       0.052\sym{**} &     (0.026)&       0.064\sym{*}  &     (0.033)\\
Round3      &       0.111\sym{***}&     (0.030)&       0.108\sym{***}&     (0.030)&       0.116\sym{***}&     (0.038)\\
GSOMSB    &       1.088\sym{***}&     (0.097)&                     &            &                     &            \\
Risk preferences&      -0.120\sym{***}&     (0.027)&      -0.118\sym{***}&     (0.027)&      -0.165\sym{***}&     (0.037)\\
White       &      -0.058         &     (0.060)&      -0.062         &     (0.061)&      -0.044         &     (0.075)\\
Middle Income&      -0.045         &     (0.062)&      -0.055         &     (0.064)&      -0.062         &     (0.077)\\
High Income &       0.065         &     (0.116)&       0.067         &     (0.117)&      -0.150         &     (0.157)\\
Midwest     &      -0.186\sym{**} &     (0.082)&      -0.197\sym{**} &     (0.084)&      -0.263\sym{**} &     (0.111)\\
South       &      -0.094         &     (0.074)&      -0.082         &     (0.076)&      -0.130         &     (0.101)\\
West        &      -0.158\sym{*}  &     (0.084)&      -0.151\sym{*}  &     (0.086)&      -0.155         &     (0.115)\\
Female      &       0.061         &     (0.055)&       0.049         &     (0.056)&       0.156\sym{**} &     (0.067)\\
Bachelor    &      -0.084         &     (0.059)&      -0.098         &     (0.061)&       0.067         &     (0.078)\\
No Children &       0.004         &     (0.060)&      -0.003         &     (0.061)&      -0.107         &     (0.081)\\
Married     &      -0.086         &     (0.061)&      -0.103\sym{*}  &     (0.062)&       0.025         &     (0.078)\\
Age         &       0.010\sym{***}&     (0.002)&       0.010\sym{***}&     (0.002)&       0.010\sym{***}&     (0.002)\\
$\sigma_u$     &       1.064\sym{***}&     (0.019)&       1.087\sym{***}&     (0.019)&       1.007\sym{***}&     (0.028)\\
$\sigma_e$    &       0.835\sym{***}&     (0.019)&       0.834\sym{***}&     (0.019)&       0.841\sym{***}&     (0.023)\\
\hline
\(n\) (observations)       &        7110         &            &        6036         &            &        4776         &            \\
\(N\) (subjects)       &        1185         &            &       1006         &            &        796        &            \\
\hline\hline
\end{tabular}
}


\begin{tablenotes}
            \footnotesize
            \item Standard errors in parentheses. * p$<$0.1, ** p$<$0.05, *** p$<$0.01.
            \item \textit{Note:} $\sigma_u$ and $\sigma_e$ denote the standard deviations of the random intercept and residual error, respectively.
            \item Notes: GSO stands for the Game Structure Obvious mechanism treatment; BDM stands for the Becker-DeGroot-Marschak mechanism treatment; Real and Hypo stand for the real and hypothetical incentives treatments, respectively.
           \item $\times$: Interaction.
           \item GSOMSB: A dummy variable for being GSO and with MS.
           \item GSOMSB: A dummy variable for being GSO and with no MS.
           \item Baseline: BDM Real, represents the baseline treatment group for the right hand-side.
           GSO Real represents the Baseline treatment group for the left-hand side.
        \end{tablenotes}

\end{table}

\clearpage

\section{Figures}

\begin{figure}[H]
    \centering
  \includegraphics[width=15cm, height=20cm]{Model50.pdf}
    \caption{ Hypothetical bias (HB) in BDM and GSO}
    \label{fig:margin_hypo}
\end{figure}





\begin{figure}[H]
    \centering
     \includegraphics[width=15cm, height=25cm]{Hypo_bias_quarBDMvsGSO_graph.pdf}
    \caption{Learning and Hypothetical bias (HB)}
    \label{fig:Learning}
\end{figure}

\end{document}